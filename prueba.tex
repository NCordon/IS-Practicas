
\usepackage{float}
\usepackage{caption}


\newcommand{\imagen}[4]{
  \begin{minipage}{\linewidth}
    \centering
    \includegraphics[width=#4\textwidth]{#1}
    \captionof{figure}{#2}
    \label{#3}
  \end{minipage} 
}

\section{Cuestión 12}
\textbf{Muestre cómo ha quedado el disco particionado una vez el sistema está instalado}

 El disco ha quedado como indica \ref{partitions}
 
 \imagen{./images/lsblk.png}{Particiones en Ubuntu Server}{partitions}{1}


\section{Cuestión 13}
\textbf{¿Cómo ha hecho el disco 2 “arrancable”? ¿Qué hace el comando \texttt{grub-install}?}
Haciendo \texttt{grub-install /dev/sdb} desde una terminal linux. Este comando instala el cargador de arranque \texttt{grub}
en la partición \texttt{/dev/sdb}. De esta forma, cuando arrancamos desde \texttt{/dev/sdb} el ordenador puede reconocer las
particiones y hacer que el usuario escoja desde qué partición iniciar.

