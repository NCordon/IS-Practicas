%%%%%%%%%%%%%%%
% This example shows the biblatex customization and footnote citation
% approach discussed at
%
%   www.khirevich.com/latex/bibliography
%   www.khirevich.com/latex/biblatex
%   www.khirevich.com/latex/footnote_citation
%
%
% Included file "example_text.tex" uses the following citation commands:
%
%   \cite (citation number of normal size in square brackets, no cite info in footnote);
%   \superfullcite (superscript citation number, full cite info in footnote);
%   \sjcitep (superscript citation number, short cite info in footnote).
%%%%%%%%%%%%%%%

\documentclass[a4paper,11pt]{article}
% Símbolo del euro
\usepackage[gen]{eurosym}
% Codificación
\usepackage[utf8]{inputenc}
% Idioma
\usepackage[spanish]{babel} % English language/hyphenation
\selectlanguage{spanish}
% Hay que pelearse con babel-spanish para el alineamiento del punto decimal
\decimalpoint
\usepackage{dcolumn}
\newcolumntype{d}[1]{D{.}{\esperiod}{#1}}
\makeatletter
\addto\shorthandsspanish{\let\esperiod\es@period@code}
\makeatother

\usepackage{longtable}
\usepackage{tabu}
\usepackage{supertabular}

\usepackage{multicol}
\newsavebox\ltmcbox

% Para algoritmos
\usepackage{algorithm}
\usepackage{algorithmic}
\usepackage{amsthm}
% Para matrices
\usepackage{amsmath}

% Símbolos matemáticos
\usepackage{amssymb}
\let\oldemptyset\emptyset
\let\emptyset\varnothing

% Hipervínculos
\usepackage{url}

\usepackage[section]{placeins} % Para gráficas en su sección.
\usepackage{mathpazo} % Use the Palatino font
\usepackage[T1]{fontenc} % Required for accented characters
\newenvironment{allintypewriter}{\ttfamily}{\par}
\setlength{\parindent}{0pt}
\parskip=8pt
\linespread{1.05} % Change line spacing here, Palatino benefits from a slight increase by default


% Imágenes
\usepackage{graphicx}
\usepackage{wrapfig} % Allows in-line images

% Referencias
\usepackage{fncylab}
\labelformat{figure}{\textit{\figurename\space #1}}

\usepackage{hyperref}


%Basado en: http://en.wikibooks.org/wiki/LaTeX/Theorems
\usepackage{amsthm}
\newtheorem*{mydef}{Definición}
\newtheorem{mydefn}{Definición}
\newtheorem{theorem}{Teorema}
\everymath{\displaystyle} % Displaystyle por defecto



% To change level of indentation
\newenvironment{answer}{%
\begin{list}{}{%
\addtolength{\hoffset}{0cm}
}%
\item[]}{\end{list}}


\makeatletter
\renewcommand{\@listI}{\itemsep=0pt} % Reduce the space between items in the itemize and enumerate environments and the bibliography
\newcommand{\imagent}[4]{
  \begin{wrapfigure}{#4}{0.5\textwidth}
    \begin{center}
    \includegraphics[width=0.5\textwidth]{#1}
    \end{center}
    \caption{#3}
    \label{#4}
  \end{wrapfigure}
}
\newcommand{\imagen}[3]{
  \begin{figure}[here]
    \begin{center}
    \includegraphics[width=0.5\textwidth]{#1}
    \end{center}
    \caption{#2}
    \label{#3}
  \end{figure}
}


%Sections don't get numbered
\setcounter{secnumdepth}{0}

\begin{document}
\begin{titlepage}

\newcommand{\HRule}{\rule{\linewidth}{0.5mm}} % Defines a new command for the horizontal lines, change thickness here

\center % Center everything on the page
 
%----------------------------------------------------------------------------------------
%	HEADING SECTIONS
%----------------------------------------------------------------------------------------
\textsc{\LARGE Universidad de Granada}\\[1.5cm]
\textsc{\Large Ingeniería de Servidores}\\[0.5cm] 

%----------------------------------------------------------------------------------------
%	TITLE SECTION
%----------------------------------------------------------------------------------------
\bigskip
\HRule \\[0.4cm]
{ \huge \bfseries Práctica I}\\[0.4cm] % Title of your document
\HRule \\[1.5cm]
 
%----------------------------------------------------------------------------------------
%	AUTHOR SECTION
%----------------------------------------------------------------------------------------

\begin{minipage}{0.4\textwidth}
\begin{center} \large
\emph{Ignacio Cordón Castillo}\\
\end{center}
\end{minipage}

%----------------------------------------------------------------------------------------
%	LOGO SECTION
%----------------------------------------------------------------------------------------

\begin{center}
\includegraphics[width=9cm]{ugr.jpg}
\end{center}
%----------------------------------------------------------------------------------------

\vspace{\fill}% Fill the rest of the page with whitespace
\large\today
\end{titlepage}  

\newpage
\tableofcontents
\newpage
% Examples of inclussion of images
%\imagent{ugr.jpg}{Logo de prueba}{ugr}{r}
%\imagen{ugr.jpg}{Logo de prueba}{ugr2}

\section{Cuestión 1}
\textbf{¿Qué modos y tipos de “Virtualización Hardware” existen?}
\begin{answer}
 Las técnicas de virtualización hardware, usadas comúnmente para crear entornos de ejecución aislados para un 
 servidor, son:
 \begin{itemize}
  \item \textbf{Virtualización completa (\textit{full virtualization})}: permite simular el hardware necesario
  para que un SO diseñado con idéntico repertorio de instrucciones a la CPU host pueda ejecutarse en un entorno
  aislado. Ejemplos de este tipo de virtualización son VirtualBox o VMware Server.
  \item \textbf{Virtualización con apoyo hardware (\textit{hardware-assisted virtualization})}: en este tipo
  de virtualización, se provee de soporte arquitectónico que permite correr varios sistemas operativos 
  huésped de manera aislada en la misma máquina. Ejemplos de este tipo de virtualización pueden encontrarse
  en Linux KVM, VMware Fusion, \ldots
  \item \textbf{Virtualización parcial (\textit{partial virtualization})}: se simulan muchas de las características
  del sistema hardware subyacente, incluyendo los espacios de direcciones. Cada máquina virtual dispone de un
  espacio de direcciones distinto. Este método no permite virtualizar un SO al completo, pero sí ejecutar
  aplicaciones en entornos totalmente aislados. Pese a ello, este método de virtualización está ampliamente
  presente en sistemas operativos como Linux o Windows, dado que es más fácil de implementar que otros
  tipos de virtuaizaciones y permite compartir recursos entre múltiples usuarios. Adolece de un importante problema:
  si no se asignan los recursos suficientes en la virtualización, al ejecutar una aplicación que intente
  usar más recursos de los destinados a la virtualización, la aplicación fallará.
  \item \textbf{Virtualización a nivel de SO (\textit{OS level virtualization})}: en este tipo de virtualización,
  varios servidores físicos pueden ser virtualizados a nivel de SO, en la capa del kernel, de manera aislada. 
  Se crean entornos virtuales en un SO corriendo en un servidor físico desde el que se asignan recursos hardware y software.\\
  Los sistemas operativos huésped comparten la misma instancia de SO host, incluido el kernel, aunque cada máquina lo ve 
  como un sistema independiente. Ejemplos de este tipo de virtualización son Solaris Containers o Linux-VServer.
  \item \textbf{Paravirtualización (\textit{paravirtualization})}: la máquina virtual no recrea un hardware
  específico para la máquina, sino que se ofrece una API a través de la cual el SO virtualizado (para lo que
  habría que modificar su código fuente), puede hacer llamadas al hipervisor (monitor de máquinas virtuales
  que se encuentra en el SO host) para comunicarse directamente con el hardware del SO host.\\
  Requiere apoyo hardware para efectuar la asignación de los espacios de direcciones.
  
  %http://en.wikipedia.org/wiki/Hardware_virtualization
  %http://es.wikipedia.org/wiki/Virtualizaci%C3%B3n
  
 \end{itemize}
\end{answer}

\section{Cuestión 2}
\textbf{Busque en Internet ofertas de servicios de, al menos, dos proveedores de
VPS (Virtual Private Server) y compare con el precio de alquiler del servicio, con el de
uso de servidores dedicados (administrados y no administrados) de características
similares.}
\begin{answer}
 Se han escogido como proveedores para realizar el ejercicio:
 \begin{itemize}
  \item Softlayer Inc.  \url{https://www.softlayer.com}\\
   Oferta un servidor VPS no administrado, localizado en Dalas (EEUU), con 8cores a 2GHz, 8 GB de RAM, 2 discos duros(uno de 
   100 GB y otro de 400 GB), Ubuntu 14.04 LTS como SO, 5000 GB de tráfico, con una sola IP, por 249.60 \$/mes.
  
   También ofrece un servidor dedicado no administrado, localizado en Dalas (EEUU), con procesador Xeon 2650(8 cores x 2GHz), 20 MB caché,
   8 GB de RAM, 1 disco duro de 500 GB, Ubuntu 14.04 LTS como SO, y 20000 GB de tráfico, con una única IP, por 499.00 \$/mes
  \item Liquid Web Inc. \url{https://www.liquidweb.com}\\
   Tiene servidores VPS no administrados, con 8GB de RAM, 300 GB de disco duro, 4 cores, CentOS como SO, y 5 TB de tráfico mensual, a 220\$/mes.
   
   También tiene servidores dedicados no administrados, con 8 GB de RAM, 1TB de disco duro, 4 cores y CentOS como SO, y 5 TB de tráfico mensual, a 199\$/mes
  \item Host Europe Iberia S.L. \url{https://www.hosteurope.es/}\\
   Ofrece un servidor VPS administrado, con 8 GB de RAM, 600 GB de disco duro, tráfico ilimitado, CentOS como SO,
   1 IP y 1 dominio propios por 39.99 /mes.\\
   Ofrece servidores dedicados administrados, con 32 GB de RAM, 2x600 GB de disco duro, tráfico ilimitado, CentOS como
   SO, 2 IP propias y dominios ilimitados por 119,35 /mes.
 \end{itemize}
\end{answer}

\section{Cuestión 3}
\textbf{Busque dos soluciones de VMSW alternativas a las propuestas de VMWare
y Virtual Box. Explique sus principales características y diferencias con las soluciones
que vamos a emplear en clase}

VirtualBox es propiedad de la empresa Oracle Corporation, se ejecuta bajo CPUs tanto de 64 como de 32 bits, permite
virtualizar sistemas con repertorio de instrucciones de 32 y 64 bits. Está disponible para Windows, Linux, Mac OS X, Solaris
y FreeBSD, y permite virtualizar sistemas operativos arbitrarios con velocidad cercana al sistema host, soportando 
instalación de controladores nativos en el SO huésped. Tiene licencia GPL para uso particular. Es propietario en su 
versión comercial para empresas.

VMware Player está creado por la empresa WMware, se ejecuta bajo CPUs de 64 bits, puede virtualizar sistemas de arquitectura
de instrucciones de 32 y 64 bits, está disponible para los SO Windows y Linux, y tiene licencia Propietaria, pero libre
de coste para uso personal. Permite virtualizar sistemas operativos arbitrarios, soporta instalación de controladores
en los SO huésped, y permite trabajar en dichos SO con una velocidad cercana al SO host.

VirtualBox y VMware permiten efectuar virtualización hardware.

Xen es propiedad de Citrix Systems. Se ejecuta en sistemas Linux tanto de 32 como 64 bits y tiene licencia libre: GNU GPLv2.
La principal diferencia con WMware y VirtualBox es que no permite la ejecución de sistemas operativos arbitrarios, 
y que permite virtualización hardware, paravirtualización, con velocidades que pueden llegar a ser equivalentes
a las del sistema host. Es compatible con la instalación de drivers en la máquina virtualizada.

KVM es propiedad de Red Hat, permite crear y se ejecuta bajo sistemas de 32 y 64 bits. Está disponible para Linux y FreeBSD
y tiene licencia libre GPL v2. Posibilita efectuar virtualización hardware, pero sólo en máquinas
que soporten virtualizaciones de los tipos específicos AMD-V y VT-x (no todos los procesadores soportan este tipo
de virtualización), tiene soporte para la instalación de drivers en los sistemas virtualizados, a los que 
permite correr bajo velocidades saproximadas al sistema host.

% http://en.wikipedia.org/wiki/Comparison_of_platform_virtualization_software

\end{document}