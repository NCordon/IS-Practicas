%%%%%%%%%%%%%%%
% This example shows the biblatex customization and footnote citation
% approach discussed at
%
%   www.khirevich.com/latex/bibliography
%   www.khirevich.com/latex/biblatex
%   www.khirevich.com/latex/footnote_citation
%
%
% Included file "example_text.tex" uses the following citation commands:
%
%   \cite (citation number of normal size in square brackets, no cite info in footnote);
%   \superfullcite (superscript citation number, full cite info in footnote);
%   \sjcitep (superscript citation number, short cite info in footnote).
%%%%%%%%%%%%%%%

\documentclass[a4paper,11pt]{article}

\usepackage[utf8]{inputenc}
% sudo apt-get install texlive-lang-spanish
\usepackage[spanish]{babel} % English language/hyphenation
\selectlanguage{spanish}
% Hay que pelearse con babel-spanish para el alineamiento del punto decimal
\decimalpoint
\usepackage{dcolumn}
\newcolumntype{d}[1]{D{.}{\esperiod}{#1}}
\makeatletter
\addto\shorthandsspanish{\let\esperiod\es@period@code}
\makeatother

\usepackage{longtable}
\usepackage{tabu}
\usepackage{supertabular}

\usepackage{multicol}
\newsavebox\ltmcbox

% Para algoritmos
\usepackage{algorithm}
\usepackage{algorithmic}
\usepackage{amsthm}
% Para matrices
\usepackage{amsmath}

% Símbolos matemáticos
\usepackage{amssymb}
\let\oldemptyset\emptyset
\let\emptyset\varnothing

% Hipervínculos
\usepackage{url}

\usepackage[section]{placeins} % Para gráficas en su sección.
\usepackage{mathpazo} % Use the Palatino font
\usepackage[T1]{fontenc} % Required for accented characters
\newenvironment{allintypewriter}{\ttfamily}{\par}
\setlength{\parindent}{0pt}
\parskip=8pt
\linespread{1.05} % Change line spacing here, Palatino benefits from a slight increase by default


% Imágenes
\usepackage{graphicx}
\usepackage{wrapfig} % Allows in-line images

% Referencias
\usepackage{fncylab}
\labelformat{figure}{\textit{\figurename\space #1}}


\usepackage{charter} % optional: activate bitstream charter font

\usepackage[hyperref=true,
            url=false,
            isbn=false,
            backref=true,
            style=custom-numeric-comp,
            %citereset=chapter,
            maxcitenames=3,
            maxbibnames=100,
            backend=bibtex, % while checking on one of my (newest) systems, this option was needed to generate bibliography
            block=none]{biblatex}

\usepackage{hyperref}

% back reference text preceding the page number ("see p.")
\DefineBibliographyStrings{english}{%
    backrefpage  = {see p.}, % for single page number
    backrefpages = {see pp.} % for multiple page numbers
}

% the followings activate 'custom-english-ordinal-sscript.lbx'
% in order to print ordinal 'edition' suffixes as superscripts,
% and adjusts (reduces) spacing between suffix and following "ed."
\DeclareLanguageMapping{english}{custom-english-ordinal-sscript}
\DeclareFieldFormat{edition}%
                   {\ifinteger{#1}%
                    {\mkbibordedition{#1}\addthinspace{}ed.}%
                    {#1\isdot}}

% removes period at the very end of bibliographic record
\renewcommand{\finentrypunct}{}

% removes period after DOI and suppresses capitalization
% of the word following DOI ("See p. xx" -> "see p. xx")
\renewcommand{\newunitpunct}{\addspace\midsentence}

\DeclareFieldFormat{journaltitle}{\mkbibemph{#1},} % italic journal title with comma
\DeclareFieldFormat[inbook,thesis]{title}{\mkbibemph{#1}\addperiod} % italic title with period
\DeclareFieldFormat[article]{title}{#1} % title of journal article is printed as normal text
\DeclareFieldFormat[article]{volume}{\textbf{#1}\addcolon\space} % makes volume of journal bold and adds colon
\DeclareFieldFormat{pages}{#1} % removes pagination (p./pp.) before page numbers

%%%%%%%%%
% the command \sjcitep defined below prints footnote citation above punctuation
\newlength{\spc} % declare a variable to save spacing value
\newcommand{\sjcitep}[2][]{% new command with two arguments: optional (#1) and mandatory (#2)
        \settowidth{\spc}{#1}% set value of \spc variable to the width of #1 argument
        \addtolength{\spc}{-1.8\spc}% subtract from \spc about two (1.8) of its values making its magnitude negative
        #1% print the optional argument
        \hspace*{\spc}% print an additional negative spacing stored in \spc after #1
        \supershortnotecite{#2}}% print (cite) the mandatory argument
%%%%%%%%%



%Basado en: http://en.wikibooks.org/wiki/LaTeX/Theorems
\usepackage{amsthm}
\newtheorem*{mydef}{Definición}
\newtheorem{mydefn}{Definición}
\newtheorem{theorem}{Teorema}
\everymath{\displaystyle} % Displaystyle por defecto



% To change level of indentation
\newenvironment{answer}{%
\begin{list}{}{%
\addtolength{\hoffset}{2cm}
}%
\item[]}{\end{list}}


\makeatletter
\renewcommand{\@listI}{\itemsep=0pt} % Reduce the space between items in the itemize and enumerate environments and the bibliography
\newcommand{\imagent}[4]{
  \begin{wrapfigure}{#4}{0.5\textwidth}
    \begin{center}
    \includegraphics[width=0.5\textwidth]{#1}
    \end{center}
    \caption{#3}
    \label{#4}
  \end{wrapfigure}
}
\newcommand{\imagen}[3]{
  \begin{figure}[here]
    \begin{center}
    \includegraphics[width=0.5\textwidth]{#1}
    \end{center}
    \caption{#2}
    \label{#3}
  \end{figure}
}

\bibliography{references}  % includes file "example_ref_list.bib" with data on the cited references
\begin{document}
\begin{titlepage}

\newcommand{\HRule}{\rule{\linewidth}{0.5mm}} % Defines a new command for the horizontal lines, change thickness here

\center % Center everything on the page
 
%----------------------------------------------------------------------------------------
%	HEADING SECTIONS
%----------------------------------------------------------------------------------------
\textsc{\LARGE Universidad de Granada}\\[1.5cm]
\textsc{\Large Ingeniería de Servidores}\\[0.5cm] 

%----------------------------------------------------------------------------------------
%	TITLE SECTION
%----------------------------------------------------------------------------------------
\bigskip
\HRule \\[0.4cm]
{ \huge \bfseries Práctica I}\\[0.4cm] % Title of your document
\HRule \\[1.5cm]
 
%----------------------------------------------------------------------------------------
%	AUTHOR SECTION
%----------------------------------------------------------------------------------------

\begin{minipage}{0.4\textwidth}
\begin{center} \large
\emph{Ignacio Cordón Castillo}\\
\end{center}
\end{minipage}

%----------------------------------------------------------------------------------------
%	LOGO SECTION
%----------------------------------------------------------------------------------------

\begin{center}
\includegraphics[width=9cm]{ugr.jpg}
\end{center}
%----------------------------------------------------------------------------------------

\vspace{\fill}% Fill the rest of the page with whitespace
\large\today
\end{titlepage}  

\newpage
\tableofcontents
\newpage
% Examples of inclussion of images
%\imagent{ugr.jpg}{Logo de prueba}{ugr}{r}
%\imagen{ugr.jpg}{Logo de prueba}{ugr2}

\section*{Cuestión 1}
\textbf{¿Qué modos y tipos de “Virtualización Hardware” existen?}
\begin{answer}
 Las técnicas de virtualización hardware, usadas comúnmente para crear entornos de ejecución aislados para un 
 servidor, son:
 \begin{itemize}
  \item \textbf{Virtualización completa (\textit{full virtualization})}: permite simular el hardware necesario
  para que un SO diseñado con idéntico repertorio de instrucciones a la CPU host pueda ejecutarse en un entorno
  aislado. Ejemplos de este tipo de virtualización son VirtualBox o VMware Server.
  \item \textbf{Virtualización con apoyo hardware (\textit{hardware-assisted virtualization})}: en este tipo
  de virtualización, se provee de soporte arquitectónico que permite correr varios sistemas operativos 
  huésped de manera aislada en la misma máquina. Ejemplos de este tipo de virtualización pueden encontrarse
  en Linux KVM, VMware Fusion, \ldots
  \item \textbf{Virtualización parcial (\textit{partial virtualization})}: se simulan muchas de las características
  del sistema hardware subyacente, incluyendo los espacios de direcciones. Cada máquina virtual dispone de un
  espacio de direcciones distinto. Este método no permite virtualizar un SO al completo, pero sí ejecutar
  aplicaciones en entornos totalmente aislados. Pese a ello, este método de virtualización está ampliamente
  presente en sistemas operativos como Linux o Windows, dado que es más fácil de implementar que otros
  tipos de virtuaizaciones y permite compartir recursos entre múltiples usuarios. Adolece de un importante problema:
  si no se asignan los recursos suficientes en la virtualización, al ejecutar una aplicación que intente
  usar más recursos de los destinados a la virtualización, la aplicación fallará.
  \item \textbf{Virtualización a nivel de SO (\textit{OS level virtualization})}: en este tipo de virtualización,
  varios servidores físicos pueden ser virtualizados a nivel de SO, en la capa del kernel, de manera aislada. 
  Se crean entornos virtuales en un SO corriendo en un servidor físico desde el que se asignan recursos hardware y software.\\
  Los sistemas operativos huésped comparten la misma instancia de SO host, incluido el kernel, aunque cada máquina lo ve 
  como un sistema independiente. Ejemplos de este tipo de virtualización son Solaris Containers o Linux-VServer.
  \item \textbf{Paravirtualización (\textit{paravirtualization})}: la máquina virtual no recrea un hardware
  específico para la máquina, sino que se ofrece una API a través de la cual el SO virtualizado (para lo que
  habría que modificar su código fuente), puede hacer llamadas al hipervisor (monitor de máquinas virtuales
  que se encuentra en el SO host) para comunicarse directamente con el hardware del SO host.\\
  Requiere apoyo hardware para efectuar la asignación de los espacios de direcciones.
  
  
 \end{itemize}
\end{answer}

\section*{Cuestión 2}
\textbf{Busque en Internet ofertas de servicios de, al menos, dos proveedores de
VPS (Virtual Private Server) y compare con el precio de alquiler del servicio, con el de
uso de servidores dedicados (administrados y no administrados) de características
similares.}

% prints author names as small caps
\renewcommand{\mkbibnamefirst}[1]{\textsc{#1}}
\renewcommand{\mkbibnamelast}[1]{\textsc{#1}}
\renewcommand{\mkbibnameprefix}[1]{\textsc{#1}}
\renewcommand{\mkbibnameaffix}[1]{\textsc{#1}}
\printbibliography

\end{document}