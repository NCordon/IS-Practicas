\documentclass[a4paper,11pt]{article}
% Símbolo del euro
\usepackage[gen]{eurosym}
% Codificación
\usepackage[utf8]{inputenc}
% Idioma
\usepackage[spanish]{babel} % English language/hyphenation
\selectlanguage{spanish}
% Hay que pelearse con babel-spanish para el alineamiento del punto decimal
\decimalpoint
\usepackage{dcolumn}
\newcolumntype{d}[1]{D{.}{\esperiod}{#1}}
\makeatletter
\addto\shorthandsspanish{\let\esperiod\es@period@code}
\makeatother
% To work with bibtex
%\usepackage{natbib}
\usepackage[backend=bibtex,style=numeric,sorting=none]{biblatex}
\bibliography{references}
\usepackage{longtable}
\usepackage{tabu}
\usepackage{supertabular}

\usepackage{multicol}
\newsavebox\ltmcbox

% Para algoritmos
\usepackage{algorithm}
\usepackage{algorithmic}
\usepackage{amsthm}
% Para matrices
\usepackage{amsmath}

% Símbolos matemáticos
\usepackage{amssymb}
\let\oldemptyset\emptyset
\let\emptyset\varnothing

% Hipervínculos
\usepackage{url}

\usepackage[section]{placeins} % Para gráficas en su sección.
\usepackage{mathpazo} % Use the Palatino font
\usepackage[T1]{fontenc} % Required for accented characters
\newenvironment{allintypewriter}{\ttfamily}{\par}
\setlength{\parindent}{0pt}
\parskip=8pt
\linespread{1.05} % Change line spacing here, Palatino benefits from a slight increase by default


% Imágenes
\usepackage{graphicx}
\usepackage{wrapfig} % Allows in-line images

% Referencias
\usepackage{fncylab}
\labelformat{figure}{\textit{\figurename\space #1}}

\usepackage{hyperref}
\hypersetup{
  colorlinks   = true, %Colours links instead of ugly boxes
  urlcolor     = blue, %Colour for external hyperlinks
  linkcolor    = blue, %Colour of internal links
  citecolor   = red %Colour of citations
}

%Basado en: http://en.wikibooks.org/wiki/LaTeX/Theorems
\usepackage{amsthm}
\newtheorem*{mydef}{Definición}
\newtheorem{mydefn}{Definición}
\newtheorem{theorem}{Teorema}
\everymath{\displaystyle} % Displaystyle por defecto



% To change level of indentation
\newenvironment{answer}{%
\begin{list}{}{%
\addtolength{\hoffset}{0cm}
}%
\item[]}{\end{list}}


\makeatletter
\renewcommand{\@listI}{\itemsep=0pt} % Reduce the space between items in the itemize and enumerate environments and the bibliography
\newcommand{\imagent}[4]{
  \begin{wrapfigure}{#4}{0.7\textwidth}
    \begin{center}
    \includegraphics[width=0.7\textwidth]{#1}
    \end{center}
    \caption{#3}
    \label{#4}
  \end{wrapfigure}
}
\newcommand{\imagen}[3]{
  \begin{figure}[H]
    \begin{center}
    \includegraphics[width=0.7\textwidth]{#1}
    \end{center}
    \caption{#2}
    \label{#3}
  \end{figure}
}

%Customize enumerate tag
\usepackage{enumitem}
%Sections don't get numbered
\setcounter{secnumdepth}{0}

\begin{document}
\begin{titlepage}

\newcommand{\HRule}{\rule{\linewidth}{0.5mm}} % Defines a new command for the horizontal lines, change thickness here

\center % Center everything on the page
 
%----------------------------------------------------------------------------------------
%	HEADING SECTIONS
%----------------------------------------------------------------------------------------
\textsc{\LARGE Universidad de Granada}\\[1.5cm]
\textsc{\Large Ingeniería de Servidores}\\[0.5cm] 

%----------------------------------------------------------------------------------------
%	TITLE SECTION
%----------------------------------------------------------------------------------------
\bigskip
\HRule \\[0.4cm]
{ \huge \bfseries Práctica I}\\[0.4cm] % Title of your document
\HRule \\[1.5cm]
 
%----------------------------------------------------------------------------------------
%	AUTHOR SECTION
%----------------------------------------------------------------------------------------

\begin{minipage}{0.4\textwidth}
\begin{center} \large
\emph{Ignacio Cordón Castillo}\\
\end{center}
\end{minipage}

%----------------------------------------------------------------------------------------
%	LOGO SECTION
%----------------------------------------------------------------------------------------

\begin{center}
\includegraphics[width=9cm]{ugr.jpg}
\end{center}
%----------------------------------------------------------------------------------------

\vspace{\fill}% Fill the rest of the page with whitespace
\large\today
\end{titlepage}  

\newpage
\tableofcontents
\newpage
% Examples of inclussion of images
%\imagent{ugr.jpg}{Logo de prueba}{ugr}{r}
%\imagen{ugr.jpg}{Logo de prueba}{ugr2}

\section{Cuestión 1}
\textbf{¿Qué modos y tipos de “Virtualización Hardware” existen?}
\begin{answer}
 Las técnicas de virtualización hardware, usadas comúnmente para crear entornos de ejecución aislados para un 
 servidor, son: \cite{hardv} \cite{virt}
 \begin{itemize}
  \item \textbf{Virtualización completa (\textit{full virtualization})}: permite simular el hardware necesario
  para que un SO diseñado con idéntico repertorio de instrucciones a la CPU host pueda ejecutarse en un entorno
  aislado. Ejemplos de este tipo de virtualización son VirtualBox o VMware Server.
  \item \textbf{Virtualización con apoyo hardware (\textit{hardware-assisted virtualization})}: en este tipo
  de virtualización, se provee de soporte arquitectónico que permite correr varios sistemas operativos 
  huésped de manera aislada en la misma máquina. Ejemplos de este tipo de virtualización pueden encontrarse
  en Linux KVM, VMware Fusion, \ldots
  \item \textbf{Virtualización parcial (\textit{partial virtualization})}: se simulan muchas de las características
  del sistema hardware subyacente, incluyendo los espacios de direcciones. Cada máquina virtual dispone de un
  espacio de direcciones distinto. Este método no permite virtualizar un SO al completo, pero sí ejecutar
  aplicaciones en entornos totalmente aislados. Pese a ello, este método de virtualización está ampliamente
  presente en sistemas operativos como Linux o Windows, dado que es más fácil de implementar que otros
  tipos de virtuaizaciones y permite compartir recursos entre múltiples usuarios. Adolece de un importante problema:
  si no se asignan los recursos suficientes en la virtualización, al ejecutar una aplicación que intente
  usar más recursos de los destinados a la virtualización, la aplicación fallará.
  \item \textbf{Virtualización a nivel de SO (\textit{OS level virtualization})}: en este tipo de virtualización,
  varios servidores físicos pueden ser virtualizados a nivel de SO, en la capa del kernel, de manera aislada. 
  Se crean entornos virtuales en un SO corriendo en un servidor físico desde el que se asignan recursos hardware y software.\\
  Los sistemas operativos huésped comparten la misma instancia de SO host, incluido el kernel, aunque cada máquina lo ve 
  como un sistema independiente. Ejemplos de este tipo de virtualización son Solaris Containers o Linux-VServer.
  \item \textbf{Paravirtualización (\textit{paravirtualization})}: la máquina virtual no recrea un hardware
  específico para la máquina, sino que se ofrece una API a través de la cual el SO virtualizado (para lo que
  habría que modificar su código fuente), puede hacer llamadas al hipervisor (monitor de máquinas virtuales
  que se encuentra en el SO host) para comunicarse directamente con el hardware del SO host.\\
  Requiere apoyo hardware para efectuar la asignación de los espacios de direcciones.
  
 \end{itemize}
\end{answer}

\section{Cuestión 2}
\textbf{Busque en Internet ofertas de servicios de, al menos, dos proveedores de
VPS (Virtual Private Server) y compare con el precio de alquiler del servicio, con el de
uso de servidores dedicados (administrados y no administrados) de características
similares.}
\begin{answer}
 Se han escogido como proveedores para realizar el ejercicio:
 \begin{itemize}
  \item Softlayer Inc.  \url{https://www.softlayer.com}\\
   Oferta un servidor VPS no administrado, localizado en Dalas (EEUU), con 8cores a 2GHz, 8 GB de RAM, 2 discos duros(uno de 
   100 GB y otro de 400 GB), Ubuntu 14.04 LTS como SO, 5000 GB de tráfico, con una sola IP, por 249.60 \$/mes.
  
   También ofrece un servidor dedicado no administrado, localizado en Dalas (EEUU), con procesador Xeon 2650(8 cores x 2GHz), 20 MB caché,
   8 GB de RAM, 1 disco duro de 500 GB, Ubuntu 14.04 LTS como SO, y 20000 GB de tráfico, con una única IP, por 499.00 \$/mes
  \item Liquid Web Inc. \url{https://www.liquidweb.com}\\
   Tiene servidores VPS no administrados, con 8GB de RAM, 300 GB de disco duro, 4 cores, CentOS como SO, y 5 TB de tráfico mensual, a 220\$/mes.
   
   También tiene servidores dedicados no administrados, con 8 GB de RAM, 1TB de disco duro, 4 cores y CentOS como SO, y 5 TB de tráfico mensual, a 199\$/mes
  \item Host Europe Iberia S.L. \url{https://www.hosteurope.es/}\\
   Ofrece un servidor VPS administrado, con 8 GB de RAM, 600 GB de disco duro, tráfico ilimitado, CentOS como SO,
   1 IP y 1 dominio propios por 39.99 /mes.\\
   Ofrece servidores dedicados administrados, con 32 GB de RAM, 2x600 GB de disco duro, tráfico ilimitado, CentOS como
   SO, 2 IP propias y dominios ilimitados por 119,35 /mes.
 \end{itemize}
\end{answer}

\section{Cuestión 3}
\textbf{Busque dos soluciones de VMSW alternativas a las propuestas de VMWare
y Virtual Box. Explique sus principales características y diferencias con las soluciones
que vamos a emplear en clase}
\begin{answer}

VirtualBox es propiedad de la empresa Oracle Corporation, se ejecuta bajo CPUs tanto de 64 como de 32 bits, permite
virtualizar sistemas con repertorio de instrucciones de 32 y 64 bits. Está disponible para Windows, Linux, Mac OS X, Solaris
y FreeBSD, y permite virtualizar sistemas operativos arbitrarios con velocidad cercana al sistema host, soportando 
instalación de controladores nativos en el SO huésped. Tiene licencia GPL para uso particular. Es propietario en su 
versión comercial para empresas.

VMware Player está creado por la empresa WMware, se ejecuta bajo CPUs de 64 bits, puede virtualizar sistemas de arquitectura
de instrucciones de 32 y 64 bits, está disponible para los SO Windows y Linux, y tiene licencia Propietaria, pero libre
de coste para uso personal. Permite virtualizar sistemas operativos arbitrarios, soporta instalación de controladores
en los SO huésped, y permite trabajar en dichos SO con una velocidad cercana al SO host.

VirtualBox y VMware permiten efectuar virtualización hardware.

Xen es propiedad de Citrix Systems. Se ejecuta en sistemas Linux tanto de 32 como 64 bits y tiene licencia libre: GNU GPLv2.
La principal diferencia con WMware y VirtualBox es que no permite la ejecución de sistemas operativos arbitrarios, 
y que permite virtualización hardware, paravirtualización, con velocidades que pueden llegar a ser equivalentes
a las del sistema host. Es compatible con la instalación de drivers en la máquina virtualizada.

KVM es propiedad de Red Hat, permite crear y se ejecuta bajo sistemas de 32 y 64 bits. Está disponible para Linux y FreeBSD
y tiene licencia libre GPL v2. Posibilita efectuar virtualización hardware, pero sólo en máquinas
que soporten virtualizaciones de los tipos específicos AMD-V y VT-x (no todos los procesadores soportan este tipo
de virtualización), tiene soporte para la instalación de drivers en los sistemas virtualizados, a los que 
permite correr bajo velocidades saproximadas al sistema host.

\cite{comp}
\end{answer}

\section{Cuestión 4}
\textbf{Enumere las cinco innovaciones en Windows 2012 R2 respecto a 2008R2 que considere más importantes.}
\begin{answer}
\begin{enumerate}
 \item \cite{smbi} Windows Server 2012 permite efectuar una instalación Server Core(instalación minimal) incluyendo interfaz gráfica,
 mientras que Windows Server 2008 sólo permitía efectuar la instalación Server Core con línea de comandos.
 
 \item Se ha mejorado el Server Message Block (SMB), que tal y como lo define \cite{smb} es un 
 \textit{protocolo de compartición de archivos, impresoras, puertos seriales y abstracción de las comunicaciones entre 
 cliente y servidor}, que en la versión 3.0 presente en Windows Server 2012 incorpora mejoras significativas en cuanto a 
 rendimiento, seguridad, y añade herramientas administrativas respecto a versiones anteriores.
 
 \item \cite{dac}Windows Server 2012 proporciona Dynamic Access Control (DAC), que permite establecer un sistema automático (para
 ciertos casos, si lo desea el administrador, también puede establecerse de manera manual) para detectar y etiquetar
 datos considerados sensibles (números de identificación, números de cuentas bancarias\ldots); restringir el acceso a
 dichos datos a ciertos grupos de usuarios; auditar el acceso a ficheros; y aplicar RMS (Rights Management Services), que permite
 encriptar ficheros y restringir acceso a los mismos, de manera automática a ciertos ficheros basándose en la información contenida en ellos.
 
 \item Mejora en el sistema de virtualización Hyper-V (competencia de VMware en la virtualización en servidores),
 añadiendo la Hyper-V Replica. Los trabajos que se virtualizan con Hyper-V pueden replicarse, esto es, puede almacenarse
 una copia de seguridad del entorno de ejecución en ese momento, llevándolos a un servidor de réplicas, y pudiendo
 restaurar en cualquier momento una copia de la virtualización.
 
 \item Windows Server 2012 incluye un nuevo sistema de archivos, llamado ReFS, que hereda de NTFS, pero deja atrás
 algunas de sus características, como las cuotas de disco o la compresión de archivos, para proporcionar características
 nuevas como la verificación de datos. Además, ReFS aumenta el límite teórico de tamaño de sistema de archivos (de 16
 exabytes a 256 zetabytes).\\
 
 La referencia para esta pregunta ha sido, mayormente: \cite{techr}	
\end{enumerate}
\end{answer}

\section{Cuestión 5}
\textbf{¿Qué empresa hay detrás de Ubuntu? ¿Qué otros productos/servicios ofrece? ¿Qué es MAAS 
(\url{https://maas.ubuntu.com/})?}
\begin{answer}
La empresa que está detrás de Ubuntu es Canonical Ltd.
Se encarga de entre otras cosas: \cite{canonical} \cite{wcanonical}

\begin{itemize}
 \item Desarrollar Unity (una variante de Gnome que constituye el servidor gráfico con el que Canonical
 distribuye por defecto Ubuntu).
 \item Mantiene Ubuntu Software Center
 \item Desarrollar Juju, que permite administrar los recursos en la nube de una granja de servidores.
 \item Se encarga de mantener el sitio \url{https://launchpad.net/}, que contiene información sobre
 PPAs (repositorios de APT, el manager de instalación de paquetes de ubuntu), bugs presentes en Ubuntu
 \item Certificar la compatibilidad de Ubuntu con equipos de marcas como Dell, HP y Lenovo
 \item Construir centros de datos en la nube para empresas de telecomunicaciones.
 \item Proporcionar soporte diario a compañías como Netflix o Google
\end{itemize}

 \cite{mass} MASS (Metal As A Service) permite establecer un control sobre los servidores físicos análogo al que se tiene sobre
 máquinas virtuales. Permite establecer un control centralizado desde el que se pueden levantar o apagar nodos de servidores,
 redistribuir los recursos hardware, \ldots
\end{answer}
 
\section{Cuestión 6}
\textbf{¿Qué relación guardan las distribuciones de Linux CentOS, Fedora y RedHat Enterprise Linux? Comente las 
similitudes y diferencias que le parezcan más significativas.}
\begin{answer}
Todas tienen relación con la empresa RedHat: Fedora cuenta con el apoyo de dicha empresa, RedHat Enterprise está basado
en Fedora, y es liberado periódicamente por la empresa mencionada, y CentOS es una distribución clonada de RedHat Enterprise
(se basa en sus releases de pago).

Por tanto, RedHat Enterprise y CentOS están basados en Fedora, pero hay diferencias entre las tres.
\begin{itemize}
 \item Fedora y CentOS son desarrolladas por la comunidad, mientras que RedHat Enterprise lo es por la compañía.
 \item Fedora y CentOS son de libre distribución, mientras que RedHat es una distribución comercial.
 \item RedHat Enterprise cuenta con el soporte técnico de RedHat, mientras que las otras dos distros no, y sólo tienen
 soporte de la comunidad de usuarios.
 \item Mientras Fedora se basa en mejorar sus características con releases muy frecuentes (6 meses), RedHat está más
 enfocada a proporcionar estabilidad a sus clientes, por lo que sus releases tienen mucha menos frecuencia, y por tanto
 las releases de CentOS tienen la misma periodicidad que las de RedHat Enterprise.
\end{itemize}

\cite{redhat}
\end{answer}

\section{Cuestión 7}
\textbf{Busque indicadores de porcentaje de uso global o de cuota de mercado de SO de Servidores. No olvide poner 
la fuente de donde saca la información y preste atención a la fecha de ésta}
\begin{answer}
  
Según \cite{quota}, donde se proporciona información de una encuesta realizada por \href{http://w3techs.com/}{W3Techs} a fecha de Septiembre de 2014:
\begin{itemize}
  \item Linux (Debian, Ubuntu, CentOS, Red Hat Enterprise Linux, Gentoo) es usado en el 36.72\% de los servidores.
  \item Otros SOs Unix (FreeBSD, HP-UX, Solaris, OS X Server) son usados en el 30.18\% de los servidores.
  \item Windows Server (2003, 2008, 2012) tiene una cuota de mercado del 33.10\%.
\end{itemize}
\end{answer}

\section{Cuestión 8}
\textbf{a) ¿De qué es el acrónimo RAID? b) ¿Qué tipos de RAID hay? c) ¿Qué diferencia hay entre RAID mediante SW y 
mediante HW?}
\begin{answer}
\begin{enumerate}[label=\alph*.]
\item Originalmente RAID era acrónimo de Redundant Array of Inexpensive disks, aunque por intereses comerciales
(Inexpensive = de bajo coste) se cambió el acrónimo a Redundant Array of Independet disks.
\item Existen, entre otros, los siguientes tipos de RAID: \cite{raid} \cite{imgsraid}
  \begin{itemize}
    \item \underline{RAID 0(Data Striping)}: Los datos se dividen y se envían sus partes a varios discos (al menos dos). De esta manera, puede paralelizarse
    la lectura de ficheros, aunque si falla alguno de los discos, se pierden todos los datos del fichero. Para ilustrar
    este sistema, en \ref{raid0} se observa cómo se almacenan 6 bloques de un archivo en los dos discos.
    \imagen{raid-0.png}{Ejemplo de RAID 0}{raid0}
    
    \item \underline{RAID 1(Mirroring)}: Se almacenan dos copias para cada datos, que se envían a dos discos distintos(el que almacena
    la copia de seguridad se llama disco espejo). Este sistema permite leer de cualquiera de los dos discos que contienen
    un dato, pudiendo paralelizarse también la lectura, pero es muy costoso en tanto que se necesitan el doble de discos
    para almacenar un dato que en RAID 0. Siendo A,B,C bloques de un fichero, se ilustra el funcionamiento de este
    tipo de RAID en \ref{raid1}
    \imagen{raid-1.png}{Ejemplo de RAID 1}{raid1}
    \item \underline{RAID 2}: no usado en la práctica, sólo constituye un modelo teórico. Divide los bits de los bloques de datos y
    los reparte entre los distintos discos, almacenándose bits de comprobación en otro/s disco/s. Para la detección de errores
    se usa un código de detección llamado paridad de Hamming. El coste de su implementación es muy elevado, por eso este sistema
    no se usa.
    
    \item \underline{RAID 3}: hay un disco que se usa para almacenar datos de paridades. Los bloques se dividen en tantos trozos
    como discos(sin contar el de paridad, que almacenará un valor para poder comprobar cuando se recupere el bloque que 
    no se ha perdido información porque alguno de los discos haya fallado) y se almacena un trozo en cada uno de ellos.
    
    \item \underline{RAID 4}: cuando se almacena un fichero, los bloques se dividen entre los distintos discos (en este caso los bloques
    no se trocean), y se usa un disco fijo para almacenar la paridad(un valor comprobatorio de la integridad de todos los
    bloques que conforman un determinado archivo).
    
    \item \underline{RAID 5}: análogo al RAID 4, pero en este caso no hay un único disco de paridad, sino que la paridad se va
    almacenando por turnos en todos los discos (por ejemplo, cuando se almacene el primer bloque, la paridad se guardará
    en el disco 1, cuando llegue otro bloque, la paridad se guardará en el disco 2, \ldots). Ej. En \ref{raid5}, cuando
    se almacenan A y B en Disk1 y Disk2 resp., se guarda un bloque de paridad p1 en Disk3; al almacenar C y D en
    Disk1 y Disk3, se almacena un bloque de paridad en 2; al guardar E en Disk2 y F en Disk3, el bloque de paridad p3
    es guardado en Disk1.
    \imagen{raid-5.png}{Ejemplo de RAID 5}{raid5}
    
    \item \underline{RAID 6}: usa el mismo principio que RAID 5, pero se almacenan ahora dos bloques de paridad (se necesita por tanto
    un disco más que en RAID 5 para almacenar la misma cantidad de datos).
    
    \item \underline{RAID 1+0}: los bloques de un archivo se reparten entre las divisiones RAID 1(conjuntos de dos discos), donde
    en cada división cada bloque es almacenado en dos discos distintos.
  \end{itemize}
  
  \item \cite{swhwraid} El software RAID conlleva un procesamiento extra por parte de la placa madre del servidor (CPU y buses), que
  en caso de usar RAIDs con paridad puede ocasionar un sobrecoste importante sobre los recursos del sistema,
  mientras que el hardware RAID efectúa el procesamiento de los discos a nivel de dispositivos de almacenamiento, sin
  el coste para buses y CPU de la máquina que se tiene en el software RAID. Además, el RAID vía software se configura
  por el Sistema Operativo, lo que imposibilita que las particiones presentes en el RAID Puedan ser compartidas entre
  sistemas operativos.

\end{enumerate}
\end{answer}

\section{Cuestión 9}
\textbf{a) ¿Qué es LVM? b)¿Qué ventaja tiene para un servidor de gama baja? c)Si va a tener un servidor web, ¿le daría
un tamaño grande o pequeño a \texttt{/var}?}
\begin{answer}
 \begin{enumerate}[label=\alph*.]
  \item \cite{lvm} LVM(Logic Volume Manager - Gestor de Volúmenes Lógicos) es un método de asignación de espacio en disco que crea
  volúmenes lógicos, que pueden ser fácilmente redimensionados, en lugar de particiones. Estos volúmenes lógicos son
  una capa de abstracción entre el sistema de archivos y el almacenamiento físico.
  En LVM, varios volúmenes físicos conforman un grupo de volúmenes lógicos, donde un volumen físico sólo puede englobar
  a una unidad de disco. Los grupos de volúmenes lógicos se dividen en volúmenes lógicos \texttt{/home,/},\ldots. Cuando
  dichos volúmenes lógicos alcanzan su máxima capacidad, se podría dedicar espacio libre de otro volumen lógico
  adscrito al mismo grupo de volúmenes lógico y redimensionarlos.
  \item \cite{benefitslvm} El espacio en LVM no tiene porqué ser contiguo para ser asignado, lo que es muy deseable en un servidor de gama
  baja que supuestamente tendrá poca capacidad de almacenamiento. Además, si en el servidor se llenan alguno de los
  volúmenes, resulta sencillo redistribuir el espacio con LVM, frente a la necesidad de reparticionar si se están
  empleando particiones.
  \item Puesto que \texttt{/var} contiene archivos temporales, sería necesario tener en cuenta que puede aumentar su
  tamaño significativamente. Si el servidor es de gama baja, lo ideal es que esté configurado con LVM, para que si en
  algún momento hace falta asignar o quitar espacio a \texttt{/var} se pudiera hacer de manera sencilla, por lo que
  en este caso sería indiferente si se le asigna inicialmente mucho o poco espacio. Si el servidor es de gama alta, y 
  está particionado en lugar de usar LVM(si usase LVM la respuesta es la misma que para servidores de gama baja), entonces
  sería conveniente asignarle inicialmente mucho espacio, asumiendo que el servidor dispondrá de espacio más que suficiente,
  para así no tener que reparticionar en ningún momento.
 \end{enumerate}
\end{answer}

\section{Cuestión 10}
\textbf{¿Es conveniente cifrar también el volumen que contiene el espacio para swap? ¿Por qué no es posible cifrar 
el volumen en el que montaremos \texttt{/boot}?}
\begin{answer}
 Es conveniente cifrar el swap, ya que en el espacio de intercambio puede haber información de gran importancia para
 programas de una máquina, como por ejemplo contraseñas. Imaginemos que una clave que está siendo usada por un programa,
 y se halla en memoria principal. Si se necesitara liberar espacio en dicha memoria, se comenzarían a llevar páginas
 de memoria a espacio de intercambio, con lo cual, si dicha clave estuviese entre las páginas llevadas a swap, la clave
 sería totalmente visible a cualquiera que consultara la información almacenada en espacio de intercambio.
 
 \cite{boot} No es posible cifrar el volumen \texttt{/boot}. Al menos, si se cifra dicho volumen, debería existir un pequeño espacio
 en disco duro donde se almacenen las claves y el algoritmo de desencriptación; esto es debido a que cuando el sistema
 se inicia y hace un prompt pidiendo la contraseña para desencriptar la partición, ya debe tener lista la clave de
 decodificación. El mismo problema surge si ciframos \texttt{/boot}: cuando se lee \texttt{/boot} se reconocen las particiones
 y se le pregunta al usuario desde cuál quiere iniciar la máquina, pero si no podemos desencriptar dicha partición, este
 proceso no puede hacerse.
 
\end{answer}

\section{Cuestión 13}
\textbf{¿Cómo ha hecho el disco 2 “arrancable”? ¿Qué hace el comando \texttt{grub-install}?}
Haciendo \texttt{grub-install /dev/sdb} desde una terminal linux. Este comando instala el cargador de arranque \texttt{grub}
en la partición \texttt{/dev/sdb}. De esta forma, cuando arrancamos desde \texttt{/dev/sdb} el ordenador puede reconocer las
particiones y hacer que el usuario escoja desde qué partición iniciar.

\section{Cuestión 14}
\begin{answer}
\textbf{¿Cuál es la principal diferencia que hay entre las versiones Standard y Datacenter de Windows 2012?}
\cite{difstdat} La versión Estándar permite montar dos máquinas virtuales en el mismo host, como máximo, mientras que la
versión Datacenter permite montar un número ilimitado de instancias de máquinas virtuales en el mismo host.
\end{answer}


\printbibliography
\end{document}