\documentclass[a4paper,11pt]{article}
% Símbolo del euro
\usepackage[gen]{eurosym}
% Codificación
\usepackage[utf8]{inputenc}
% Idioma
\usepackage[spanish]{babel} % English language/hyphenation
\selectlanguage{spanish}
% Hay que pelearse con babel-spanish para el alineamiento del punto decimal
\decimalpoint
\usepackage{dcolumn}
\newcolumntype{d}[1]{D{.}{\esperiod}{#1}}
\makeatletter
\addto\shorthandsspanish{\let\esperiod\es@period@code}
\makeatother
% To work with bibtex
%\usepackage{natbib}
\usepackage[backend=bibtex,style=numeric,sorting=none]{biblatex}
\bibliography{references}
\usepackage{longtable}
\usepackage{tabu}
\usepackage{supertabular}

\usepackage{multicol}
\newsavebox\ltmcbox

% Para algoritmos
%\usepackage{algorithm}
%\usepackage{algorithmic}
\usepackage{amsthm}
% Para matrices
\usepackage{amsmath}

% Símbolos matemáticos
\usepackage{amssymb}
\let\oldemptyset\emptyset
\let\emptyset\varnothing

% Hipervínculos
\usepackage{url}

\usepackage[section]{placeins} % Para gráficas en su sección.
\usepackage{mathpazo} % Use the Palatino font
\usepackage[T1]{fontenc} % Required for accented characters
\newenvironment{allintypewriter}{\ttfamily}{\par}
\setlength{\parindent}{0pt}
\parskip=8pt
\linespread{1.05} % Change line spacing here, Palatino benefits from a slight increase by default


% Imágenes
\usepackage{graphicx}
\usepackage{float}
\usepackage{caption}
\usepackage{wrapfig} % Allows in-line images

% Referencias
\usepackage{fncylab}
\labelformat{figure}{\textit{\figurename\space #1}}

\usepackage{hyperref}
\hypersetup{
  colorlinks   = true, %Colours links instead of ugly boxes
  urlcolor     = blue, %Colour for external hyperlinks
  linkcolor    = blue, %Colour of internal links
  citecolor   = red %Colour of citations
}

%Basado en: http://en.wikibooks.org/wiki/LaTeX/Theorems
\usepackage{amsthm}
\newtheorem*{mydef}{Definición}
\newtheorem{mydefn}{Definición}
\newtheorem{theorem}{Teorema}
\everymath{\displaystyle} % Displaystyle por defecto

% To include code
\usepackage{xcolor}
\usepackage{listings}

% code in bash style
\lstdefinestyle{BashInputStyle}{
  language=bash,
  basicstyle=\small\ttfamily,
  numbers=left,
  numberstyle=\tiny,
  numbersep=3pt,
  columns=fullflexible,
  backgroundcolor=\color{gray!20},
  xleftmargin=0.1\linewidth
}


% To change level of indentation
\newenvironment{answer}{%
\begin{list}{}{%
}%
\item[]}{\end{list}}


\makeatletter
\renewcommand{\@listI}{\itemsep=0pt} % Reduce the space between items in the itemize and enumerate environments and the bibliography
\newcommand{\imagent}[4]{
  \begin{wrapfigure}{#4}{0.7\textwidth}
    \begin{center}
    \includegraphics[width=0.7\textwidth]{#1}
    \end{center}
    \caption{#3}
    \label{#4}
  \end{wrapfigure}
}

\newcommand{\imagen}[4]{
  \begin{minipage}{\linewidth}
    \centering
    \includegraphics[width=#4\textwidth]{#1}
    \captionof{figure}{#2}
    \label{#3}
  \end{minipage} 
}

%Customize enumerate tag
\usepackage{enumitem}
%Sections don't get numbered
\setcounter{secnumdepth}{0}

\begin{document}
%%%%%%%%%%%%%%%%%%%%%%%%%%%%%%%%%%%%%%%%%
% University Assignment Title Page 
% LaTeX Template
% Version 1.0 (27/12/12)
%
% This template has been downloaded from:
% http://www.LaTeXTemplates.com
%
% Original author:
% WikiBooks (http://en.wikibooks.org/wiki/LaTeX/Title_Creation)
% Modified by: NCordon (https://github.com/NCordon)
%
% License:
% CC BY-NC-SA 3.0 (http://creativecommons.org/licenses/by-nc-sa/3.0/)
% 
% Instructions for using this template:
% This title page is capable of being compiled as is. This is not useful for 
% including it in another document. To do this, you have two options: 
%
% 1) Copy/paste everything between \begin{document} and \end{document} 
% starting at \begin{titlepage} and paste this into another LaTeX file where you 
% want your title page.
% OR
% 2) Remove everything outside the \begin{titlepage} and \end{titlepage} and 
% move this file to the same directory as the LaTeX file you wish to add it to. 
% Then add \input{./title_page_1.tex} to your LaTeX file where you want your
% title page.
%
%%%%%%%%%%%%%%%%%%%%%%%%%%%%%%%%%%%%%%%%%
\begin{titlepage}

\newcommand{\HRule}{\rule{\linewidth}{0.5mm}} % Defines a new command for the horizontal lines, change thickness here

\center % Center everything on the page
 
%----------------------------------------------------------------------------------------
%	HEADING SECTIONS
%----------------------------------------------------------------------------------------
\textsc{\LARGE Universidad de Granada}\\[1.5cm]
\textsc{\Large Ingeniería de Servidores}\\[0.5cm] 

%----------------------------------------------------------------------------------------
%	TITLE SECTION
%----------------------------------------------------------------------------------------
\bigskip
\HRule \\[0.4cm]
{ \huge \bfseries Práctica II}\\[0.4cm] % Title of your document
\HRule \\[1.5cm]
 
%----------------------------------------------------------------------------------------
%	AUTHOR SECTION
%----------------------------------------------------------------------------------------

\begin{minipage}{0.4\textwidth}
\begin{center} \large
\emph{Ignacio Cordón Castillo}\\
\end{center}
\end{minipage}

%----------------------------------------------------------------------------------------
%	LOGO SECTION
%----------------------------------------------------------------------------------------

\begin{center}
\includegraphics[width=9cm]{../images/ugr.jpg}
\end{center}
%----------------------------------------------------------------------------------------

\vspace{\fill}% Fill the rest of the page with whitespace
\large\today
\end{titlepage}  

\newpage
\tableofcontents
\newpage
% Examples of inclussion of images
%\imagent{ugr.jpg}{Logo de prueba}{ugr}
%\imagen{ugr.jpg}{Logo de prueba}{ugr2}{size relative to the \textwidth}

\section{Cuestión 1}
\textbf{Proporcione ejemplos de llamada a a yum para buscar, instalar y eliminar paquetes (Pista: man yum)}
\begin{answer}
  Para buscar:
  \begin{lstlisting}[style=BashInputStyle]
  yum search firefox
  \end{lstlisting}
  Para instalar:
  \begin{lstlisting}[style=BashInputStyle]
  yum install firefox
  \end{lstlisting}
  Para eliminar:
  \begin{lstlisting}[style=BashInputStyle]
  yum remove firefox
  \end{lstlisting}
\end{answer}

\section{Cuestión 2}
\textbf{¿Qué ha de hacer para que yum pueda tener acceso a Internet a través de un proxy?(Pistas: archivo de configuración
en /etc, proxy: stargate.ugr.es:3128). ¿Cómo añadimos un nuevo repositorio?}
\begin{answer}
  Basta con modificar el archivo \texttt{/etc/yum.conf}, y añadir la línea:
  \begin{lstlisting}[style=BashInputStyle]
  proxy=http://stargate.ugr.es:3128
  \end{lstlisting}
  
  La referencia empleada para realizar la configuración ha sido: \cite{ej2}
\end{answer}

\section{Cuestión 3}
\textbf{Proporcione ejemplos de comandos para buscar un paquete en un repositorio y el correspondiente para instalarlo. 
(Pista: man apt-get ; man apt-cache)}
\begin{answer}
  \begin{lstlisting}[style=BashInputStyle]
  apt-cache search firefox
  \end{lstlisting}
\end{answer}

\section{Cuestión 4}
\textbf{Indiqué como debe modificar la configuración de apt para acceder a los repositorios a través del proxy. ¿Cómo 
añadimos un nuevo repositorio?}
\begin{answer}
  Para acceder a través de un proxy, es necesario modificar el archivo \texttt{/etc/apt/apt.conf},
  añadiendo la siguiente línea:
  \begin{lstlisting}[style=BashInputStyle]
  Acquire::http::Proxy "http://stargate.ugr.es:3128";
  \end{lstlisting}
  Para añadir un repositorio:
  \begin{lstlisting}[style=BashInputStyle]
  sudo add-apt-repository "url"
  \end{lstlisting}
  
  Las referencias empleadas para confeccionar esta respuesta han sido: \cite{aptproxy} y
  \cite{aptrepo}.
\end{answer}

\section{Cuestión 5}
\textbf{¿Qué diferencia hay entre telnet y ssh?}
\begin{answer}
  Se trata de protocolos que sirven al mismo uso: la conexión a servidores remotos.
  La principal diferencia entre ambos es que SSH es más seguro que telnet. SSH transmite
  los datos encriptados usando una clave pública para autenticar a la fuente de transmisión, 
  mientras que en telnet se transmiten los datos en texto plano y no hay autenticación.
  
  Ahora bien, puesto que los datos necesitan encriptarse, en SSH, los paquetes contienen
  información sobre la encriptación, con lo que la cantidad de datos que se transmiten
  en un paquete en SSH es menor que en telnet, siendo ésta otra diferencia.
  
  Telnet emplea el puerto 23 para efectuar las transmisiones y se sigue empleando
  en algunas redes privadas, mientras que SSH usa el puerto 22 y ha desplazado
  a telnet en su uso en redes tanto públicas como privadas, debido a los problemas
  de seguridad que presenta éste.
  
  Se ha empleado como referencia: \cite{sshtelnet}, \cite{linuxssh}.
\end{answer}

\section{Cuestión 6}
\textbf{Modifique la configuración de SSH para que impida el acceso remoto del usuario root y cambie el puerto por defecto. 
Indique las líneas modificadas en el fichero de configuración y ponga de manifiesto el cambio mediante capturas de 
pantallas en las que se aprecie el  comportamiento antes y después de los cambios. Tenga en cuenta que debe reiniciar 
el servicio para que tome los cambios}
\begin{answer}
  
\end{answer}

\section{Cuestión 7}
\textbf{Configure una instancia de Linux de forma que pueda acceder remotamente (desde otra instancia o desde su anfitrión) 
sin introducir contraseña (Pistas: ssh-keygen, ssh-copy-id). Documente el proceso que ha seguido indicando y explicando los
comandos utilizados así como posibles cambios en la configuración del servicio. Muestre con capturas de pantalla que puede 
conectar al servidor ssh remoto sin introducir contraseña.}
\begin{answer}
  
\end{answer}

\section{Cuestión 8}
\textbf{En muchas ocasiones es necesario reiniciar un servicio para que tome los cambios en su configuración. Indique los 
comandos que puede emplear en Ubuntu y CentOS para hacerlo.}
\begin{answer}
  
\end{answer}

\section{Cuestión 9}
\textbf{Ponga de manifiesto el funcionamiento de PHP en Apache creando un fichero php que presente su nombre y apellidos 
y accediéndolo con un navegador web. Presente la captura de pantalla del resultado. Ponga de manifiesto el funcionamiento
de MySQL accediendo a la BBDDs por defecto (mysql) y consultando los usuarios definidos en el sistema (select * from user). 
Documente con capturas de pantalla el acceso y resultado de la consulta.}
\begin{answer}
  
\end{answer}

\section{Cuestión 10}
\textbf{Para poner de manifiesto que el servidor está funcionando, acceda con un navegador web a su propio equipo 
(localhost). Cree una página HTML básica con su nombre y apellidos y publíquela en su servidor IIS. Muestre, con una 
captura de pantalla, como accede a dicha página con el navegador web.}
\begin{answer}
  
\end{answer}

\section{Cuestión 11}
\textbf{Escriba un breve contenido en un fichero de texto plano, cópielo y modifíquelo ligeramente en un segundo archivo, 
por ejemplo, añadiendo un par de líneas. Calcule las diferencias entre el fichero original y el modificado. Indique los
comandos necesarios para aplicar el parche así generado sobre el primer archivo y obtener el segundo. Documente el proceso 
con capturas de pantalla de cada paso.}
\begin{answer}
  
\end{answer}

\section{Cuestión 12}
\textbf{Realice la instalación de esta aplicación y pruebe a modificar algún parámetro de algún servicio. Muestre las 
capturas de el proceso de modificación y ponga de manifiesto el resultado.}
\begin{answer}
  
\end{answer}

\printbibliography
\end{document}