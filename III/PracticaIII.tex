\documentclass[a4paper,10pt]{article}
% Símbolo del euro
\usepackage[gen]{eurosym}
% Codificación
\usepackage[utf8]{inputenc}
% Idioma
\usepackage[spanish]{babel} % English language/hyphenation
\selectlanguage{spanish}
% Hay que pelearse con babel-spanish para el alineamiento del punto decimal
\decimalpoint
\usepackage{dcolumn}
\newcolumntype{d}[1]{D{.}{\esperiod}{#1}}
\makeatletter
\addto\shorthandsspanish{\let\esperiod\es@period@code}
\makeatother
% To work with bibtex
%\usepackage{natbib}
\usepackage[backend=bibtex,style=numeric,sorting=none]{biblatex}
\bibliography{references}
\usepackage{longtable}
\usepackage{tabu}
\usepackage{supertabular}

\usepackage{multicol}
\newsavebox\ltmcbox

% Para algoritmos
%\usepackage{algorithm}
%\usepackage{algorithmic}
\usepackage{amsthm}
% Para matrices
\usepackage{amsmath}

% Símbolos matemáticos
\usepackage{amssymb}
\let\oldemptyset\emptyset
\let\emptyset\varnothing

% Hipervínculos
\usepackage{url}

\usepackage[section]{placeins} % Para gráficas en su sección.
\usepackage{mathpazo} % Use the Palatino font
\usepackage[T1]{fontenc} % Required for accented characters
\newenvironment{allintypewriter}{\ttfamily}{\par}
\setlength{\parindent}{0pt}
\parskip=8pt
\linespread{1.05} % Change line spacing here, Palatino benefits from a slight increase by default


% Imágenes
\usepackage{graphicx}
\usepackage{float}
\usepackage{caption}
\usepackage{wrapfig} % Allows in-line images

% Referencias
\usepackage{fncylab}
\labelformat{figure}{\textit{\figurename\space #1}}

\usepackage{hyperref}
\hypersetup{
  colorlinks   = true, %Colours links instead of ugly boxes
  urlcolor     = blue, %Colour for external hyperlinks
  linkcolor    = blue, %Colour of internal links
  citecolor   = red %Colour of citations
}

%Basado en: http://en.wikibooks.org/wiki/LaTeX/Theorems
\usepackage{amsthm}
\newtheorem*{mydef}{Definición}
\newtheorem{mydefn}{Definición}
\newtheorem{theorem}{Teorema}
\everymath{\displaystyle} % Displaystyle por defecto

% To include code
\usepackage{xcolor}
\usepackage{listings}

% code in bash style
\lstdefinestyle{BashInputStyle}{
  language=bash,
  basicstyle=\ttfamily,
  numberstyle=\tiny,
  numbersep=3pt,
  columns=fullflexible,
  backgroundcolor=\color{gray!20},
  showspaces=false,               % show spaces adding particular underscores
  showstringspaces=false,
  %xleftmargin=0.1\linewidth
}

% To change level of indentation
\newenvironment{answer}{%
\begin{list}{}{%
}%
\item[]}{\end{list}}


\makeatletter
\renewcommand{\@listI}{\itemsep=0pt} % Reduce the space between items in the itemize and enumerate environments and the bibliography
\newcommand{\imagent}[4]{
  \begin{wrapfigure}{#4}{0.7\textwidth}
    \begin{center}
    \includegraphics[width=0.7\textwidth]{#1}
    \end{center}
    \caption{#3}
    \label{#4}
  \end{wrapfigure}
}

\newcommand{\imagen}[4]{
  \begin{minipage}{\linewidth}
    \centering
    \includegraphics[width=#4\textwidth]{#1}
    \captionof{figure}{#2}
    \label{#3}
  \end{minipage} 
}

%Customize enumerate tag
\usepackage{enumitem}
%Sections don't get numbered
\setcounter{secnumdepth}{0}
\usepackage{fontspec}
\setmainfont{Arial}
\usepackage{geometry}
 \geometry{
 a4paper,
 total={210mm,297mm},
 left=30mm,
 right=30mm,
 top=25mm,
 bottom=25mm,
 }
 
\begin{document}
%%%%%%%%%%%%%%%%%%%%%%%%%%%%%%%%%%%%%%%%%
% University Assignment Title Page 
% LaTeX Template
% Version 1.0 (27/12/12)
%
% This template has been downloaded from:
% http://www.LaTeXTemplates.com
%
% Original author:
% WikiBooks (http://en.wikibooks.org/wiki/LaTeX/Title_Creation)
% Modified by: NCordon (https://github.com/NCordon)
%
% License:
% CC BY-NC-SA 3.0 (http://creativecommons.org/licenses/by-nc-sa/3.0/)
% 
% Instructions for using this template:
% This title page is capable of being compiled as is. This is not useful for 
% including it in another document. To do this, you have two options: 
%
% 1) Copy/paste everything between \begin{document} and \end{document} 
% starting at \begin{titlepage} and paste this into another LaTeX file where you 
% want your title page.
% OR
% 2) Remove everything outside the \begin{titlepage} and \end{titlepage} and 
% move this file to the same directory as the LaTeX file you wish to add it to. 
% Then add \input{./title_page_1.tex} to your LaTeX file where you want your
% title page.
%
%%%%%%%%%%%%%%%%%%%%%%%%%%%%%%%%%%%%%%%%%
\begin{titlepage}

\newcommand{\HRule}{\rule{\linewidth}{0.5mm}} % Defines a new command for the horizontal lines, change thickness here

\center % Center everything on the page
 
%----------------------------------------------------------------------------------------
%	HEADING SECTIONS
%----------------------------------------------------------------------------------------
\textsc{\LARGE Universidad de Granada}\\[1.5cm]
\textsc{\Large Ingeniería de Servidores}\\[0.5cm] 

%----------------------------------------------------------------------------------------
%	TITLE SECTION
%----------------------------------------------------------------------------------------
\bigskip
\HRule \\[0.4cm]
{ \huge \bfseries Práctica III}\\[0.4cm] % Title of your document
\HRule \\[1.5cm]
 
%----------------------------------------------------------------------------------------
%	AUTHOR SECTION
%----------------------------------------------------------------------------------------

\begin{minipage}{0.4\textwidth}
\begin{center} \large
\emph{Ignacio Cordón Castillo}\\
\end{center}
\end{minipage}

%----------------------------------------------------------------------------------------
%	LOGO SECTION
%----------------------------------------------------------------------------------------

\begin{center}
\includegraphics[width=9cm]{../images/ugr.jpg}
\end{center}
%----------------------------------------------------------------------------------------

\vspace{\fill}% Fill the rest of the page with whitespace
\large\today
\end{titlepage}  

\newpage
\tableofcontents
\newpage
% Examples of inclussion of images
%\imagent{ugr.jpg}{Logo de prueba}{ugr}
%\imagen{ugr.jpg}{Logo de prueba}{ugr2}{size relative to the \textwidth}

\section{Cuestión 1}
\textbf{\begin{enumerate}[label=\alph*.]
         \item En qué archivos se guarda registro de  los paquetes instalados en sistema con los gestores de paquetes de Ubuntu 
         y centOS. Durante la práctica 2 instaló LAMP como un único paquete o instalando cada componente diferenciado. Busque
         en el archivo de registro las líneas correspondientes a la instalación y preséntalas.         
         \item En el directorio /var/log es común encontar archivos con extensiones en formato <numero>.gz.  Por  ejemplo,  
         .1.gz.  Explique  como  se  generan  estos  archivos  y  que relación guardan entre ellos
        \end{enumerate}}
\begin{answer}
 \begin{enumerate}[label=\alph*.] %[label=(\Alph*)] [label=(\roman*)]
  \item Para conocer los paquetes instalados en Ubuntu, basta consultar el archivo de órdenes introducidas en \texttt{apt-get},
  que se guarda en \texttt{/var/log/apt/history.log}, con la información
  de los paquetes que se instalaron, actualizaron o borraron, producto de ejecutar dichas órdenes.
  
  Ahora bien, como \texttt{apt-get} es un front-end (una intrerfaz) de \texttt{dpkg} en las distribuciones Debian, también
  podemos consultar el histórico de \texttt{dpkg} en \texttt{/var/log/dpkg.log}, que incluirá los paquetes instalados a través
  de \texttt{apt-get}, aunque también los instalados directamente a través de \texttt{dpkg}
  \cite{apt}
  
  El archivo que alamacena el histórico de paquetes instalados en CentOS, esto es, a través de \texttt{yum}, se encuentra en:
  \begin{center}\texttt{/var/log/yum.log}\end{center}
  \cite{yum}
  
  Las líneas correspondientes a la instalación de LAMP son las correspondientes a \ref{lamp1} y \ref{lamp2}
  
  \imagen{../images/11p3.jpeg}{Instalación de \texttt{apache}}{lamp1}{1.0}

  \imagen{../images/12p3.jpeg}{Instalación de \texttt{mysql} y \texttt{php}}{lamp2}{1.0}

  
  
  % Falta paquetes de LAMP
  \item Los archivos del tipo \texttt{daemon.log.1.gz}, \texttt{daemon.log.2.gz}, \ldots\\ corresponden a antiguos logs del comando
  \texttt{daemon} comprimidos. Análogamente, esto es extensible a los históricos de cualquier otro comando. Cada cierto tiempo, el archivo
  \texttt{.log} se renombra a un archivo de la forma \texttt{.log.<numero>} donde se van numerando de forma ascendente (a un número
  mayor, más reciente es el log), y el comando comenzará un nuevo histórico desde dicho momento. Pasado cierto tiempo, se comprimen dichos archivos, generando  \texttt{daemon.log.1.gz}, \texttt{daemon.log.2.gz}, \ldots; 
  de éste modo, los archivos ocupan menos espacio en disco duro, y aún están disponibles para su consulta.
  A este comportamiento, se le llama \textbf{Rotación de Logs}, y es el comando \texttt{logrotate}, programado a través del demonio
  \texttt{cron} quien se encarga de llevar a cabo dicha tarea de ir renombrando y comprimiento logs. Asimismo, dicho programa
  se puede encargar de crear nuevos logs una vez renombrados los antiguos, en función de si está especificado en su configuración
  o no. El archivo de configuración está disponible en \texttt{/etc/logrotate.conf}
  \cite{logs}
 \end{enumerate}

\end{answer}

\section{Cuestión 2}
\textbf{Dentro de /proc existe un archivo para el estado de los multidevice como es el caso de
nuestro /dev/md0 creado en la P1.\\
Vamos a monitorizar el proceso de sustituir un disco dañado por uno nuevo y cómo
podemos saber cuándo el sistema está preparado para continuar operando con
normalidad.\\
Indique los pasos que ha seguido, comandos empleados y significado de los mismos. Junto a cada comando, presente las líneas
del RAID que son significativas en cada paso: indicación de fallo, reemplazo, inicio y finalización de la reconstrucción del RAID.}
\begin{answer}
 Antes de comenzar el proceso, introducimos un disco de 8GB en la máquina virtual (del tamaño de los dos ya presentes).
antes de iniciarla, ya que durante su ejecución no podemos ni añadir ni retirar discos.
En nuestro caso, tal y como releja la \ref{rprevprev} tiene por identificador \texttt{/dev/sdc}
 
\imagen{../images/31p3.png}{Se muestra \texttt{/dev/sdc} recién introducido}{rprevprev}{1.0}

 Tenemos abiertas dos terminales en la máquina virtual. En una, antes de comenzar todo el proceso, tecleamos:
\begin{lstlisting}[style=BashInputStyle]
 watch cat /proc/mdstat
\end{lstlisting}

Tal y como muestra dicha terminal al comienzo están los dos discos duros (\texttt{/dev/sda1} y \texttt{/dev/sdb1}) en correcto funcionamiento en el
RAID (en nuestro caso \texttt{md0} tal y como se indica en la \ref{rprev}), esto es, están marcados como activos.

\imagen{../images/32p3.png}{Estado de /proc/mdstat antes de modificar el RAID}{rprev}{1.0}

En la otra terminal tecleamos lo reflejado en la \ref{r1}. Esto marca el disco \texttt{/dev/sda1} como defectuoso
dentro del RAID, y por tanto la próxima vez que se reinicie, el disco ya no formará parte del RAID.

\imagen{../images/32p3.png}{Se marca el disco \texttt{/dev/sda1} como defectuoso}{r1}{1.0}

\texttt{/dev/sda1[2] (F)} se muestra en \texttt{/proc/mdstat} indicando que el disco ha fallado. Se refleja en la
\ref{r2}

\imagen{../images/34p3.png}{\texttt{/proc/mdsat} ya marca el disco como defectuoso}{r2}{1.0}

Hacemos:
\begin{lstlisting}[style=BashInputStyle]
 sudo mdadm --manage /dev/md0 --add /dev/sdc
\end{lstlisting}
tal y como se refleja en la \ref{r3}. Esto añade el nuevo disco al RAID.

\imagen{../images/35p3.png}{Añadimos el nuevo disco al RAID}{r3}{1.0}

Comienza la reconstrucción de \texttt{/dev/sdc}, tal y como se refleja en \ref{r4}. Esto es, debe ser un espejo
de \texttt{/dev/sdb1}, y por tanto se duplica el contenido de este último en el nuevo disco.

\imagen{../images/36p3.png}{Comienza su reconstrucción}{r4}{1.0}

Cuando finaliza su reconstrucción, el disco ya está utilizable, tal y como muestra \ref{r5}; \texttt{/dev/sda1} sigue marcado como defectuoso,
y se eliminará del RAID cuando se reinicie la máquina virtual, quedando sólo \texttt{/dev/sdb1} y \texttt{/dev/sdc}
formando parte del RAID.

\imagen{../images/37p3.png}{Termina su reconstrucción}{r5}{1.0}
\end{answer}

\section{Cuestión 3}
\textbf{ Añada a la configuración de cron una tarea que se ejecute diariamente y que copie una vez al día el contenido del directorio
\~/codigo a \~/seguridad\$fecha donde \$fecha es la fecha actual del sistema (puede usar el comando
date). Otra tarea, se ejecutará una vez al mes y reunirá todos los directorios diarios creados para el mes pasado
en un archivo \~/seguridad/dirCodigo.<numero>.gz. Presente las líneas de configuración de cron afectadas, explicando
su significado. Si crea ficheros por lotes, presente y explique el código.}
\begin{answer}
 Creamos en una carpeta, por ejemplo en \texttt{\$HOME} el siguiente script, de nombre \texttt{diario.sh}, y le damos permisos de ejecución.
 
\begin{lstlisting}[style=BashInputStyle]
#!/bin/bash

if [[ -d ~/codigo ]]
then
    DIR=~/seguridad/$(date +%m-%d-%y)
    mkdir -p $DIR 2> /dev/null
    cp ~/codigo/* $DIR 2> /dev/null
else
    mkdir -p ~/codigo
fi
\end{lstlisting}
  
  El script comprueba primero si existe el directorio \texttt{~/codigo}. Si no existe, lo crea. Si existe, crea una carpeta
  con el formato \texttt{mes-dia-año} de la fecha en que nos encontremos en \texttt{~/seguridad}, y copia el contenido de \texttt{~/codigo}
  en ella.
  
  Creamos en \texttt{\$HOME} también el siguiente script, de nombre \texttt{semanal.sh}:
  
\begin{lstlisting}[style=BashInputStyle]
#!/bin/bash

let MONTH=$(date +%m)-1
N=0

if [[ MONTH -lt 10 ]]
then
    MONTH=0$MONTH
fi


for FILE in $(ls ~/seguridad/dirCodigo*.gz 2> /dev/null)
do
    ACTUAL=$(echo $FILE | egrep -o "[[:digit:]]+")
    let ACTUAL=$ACTUAL+1
    
    if [[ ACTUAL -gt N ]]
    then
	N=$ACTUAL
    fi
done

tar -czvf ~/seguridad/dirCodigo$N.gz ~/seguridad/${MONTH}* &> /dev/null
\end{lstlisting}
  
  El script halla el mes actual, y le resta uno. Como esta operación hará que el mes pierda el 0 inicial si lo llevaba, se lo añadimos
  si se trata de un mes menor a 10. Esto lo emplearemos para hallar los directorios del mes anterior posteriormente.
  El script halla a continuación el número por el que proseguir numerando los archivos \texttt{.gz}. Por ejemplo, si ya estaban
  creados los archivos \texttt{dirCodigo0.gz} y \texttt{dirCodigo1.gz}, el archivo que creará el script tendrá por nombre \texttt{dirCodigo2.gz}
  Por último el script empaqueta en el archivo \texttt{.gz} de número hallado anteriormente todos los directorios del mes anterior.

  Ahora añadimos las siguientes líneas a \texttt{/etc/crontab}:
  \begin{lstlisting}[style=BashInputStyle]
   0 12 * * * $HOME/diario.sh
   0 12 * * 0 $HOME/semanal.sh
  \end{lstlisting}
  
  La primera línea hace que el script \texttt{diario.sh} se ejecute a las 12:00 AM  todos los días.
  La segunda hace que el script \texttt{semanal.sh} se ejecute a las 12:00 AM todos los domingos.
  
  Otra alternativa es usar \texttt{anacron} en lugar de \texttt{cron}, para lo que solamente habría que copiar
  los ficheros \texttt{diario.sh} y \texttt{semanal.sh} a \texttt{/etc/cron.daily} y \texttt{/etc/cron.weekly} respectivamente.
  La diferencia fundamental con \texttt{cron} es que \texttt{anacron} no espera que el equipo esté encendido continuamente, mientras que
  \texttt{cron} ejecuta las tareas a una hora prefijada y si el equipo no está encendido a esa hora, ya no vuelve a intentar ejecutar la tarea
  ese día, aunque el equipo esté operativo más tarde.
  
  \cite{cron} \cite{anacron}
  \end{answer}

\section{Cuestión 4}
\textbf{\begin{enumerate}
         \item Pruebe a ejecutar el comando \texttt{dmesg}, conectar un dispositivo USB y vuelva a ejecutar el comando. Copie y peque las líneas
         que hacen mención al dispositivo conectado (considere usar dmesg|tail).
         %??????????
         \item Explique las diferencias(si las hay) entre o consueltar el consultar el contenido del archivo /var/log/dmesg.
        \end{enumerate}}
\begin{answer}
 \begin{enumerate}
  \item Para la primera parte podemos usar \texttt{watch 'dmesg | tail -25'}, aunque si necesitásemos más líneas,
  podríamos emplear un número mayor que 25.
  Antes de conectar se tiene \ref{prevusb}.
  
  \imagen{../images/41p3.png}{Salida de \texttt{dmesg} previa a conectar el USB}{prevusb}{.9}
  
  Después de conectar el usb se tiene \ref{afterusb}.
  
   \imagen{../images/42p3.png}{Salida de \texttt{dmesg} posterior a conectar el USB}{afterusb}{.9}
   
   Por tanto las líneas obtenidas han sido:
  \begin{lstlisting}[style=BashInputStyle,basicstyle=\footnotesize\ttfamily]
[  125.859239] usb 1-1: USB disconnect, device number 2
[  127.720081] usb 1-1: new high-speed USB device number 3 using ehci-pci
[  127.880837] usb 1-1: New USB device found, idVendor=8564, idProduct=1000
[  127.880841] usb 1-1: New USB device strings: Mfr=1, Product=2, SerialNumber=3
[  127.880843] usb 1-1: Product: Mass Storage Device
[  127.880845] usb 1-1: Manufacturer: JetFlash
[  127.880846] usb 1-1: SerialNumber: XZCAS49T
[  127.883114] usb-storage 1-1:1.0: USB Mass Storage device detected
[  127.883226] scsi6 : usb-storage 1-1:1.0
[  128.887776] scsi 6:0:0:0: Direct-Access     JetFlash Transcend 2GB    8.07 PQ: 0 ANSI: 2
[  128.887988] sd 6:0:0:0: Attached scsi generic sg4 type 0
[  128.912840] sd 6:0:0:0: [sdd] 3948544 512-byte logical blocks: (2.02 GB/1.88 GiB)
[  128.919565] sd 6:0:0:0: [sdd] Write Protect is off
[  128.919570] sd 6:0:0:0: [sdd] Mode Sense: 03 00 00 00
[  128.926431] sd 6:0:0:0: [sdd] No Caching mode page found
[  128.926564] sd 6:0:0:0: [sdd] Assuming drive cache: write through
[  129.422893]  sdd: sdd1
[  129.461971] sd 6:0:0:0: [sdd] Attached SCSI removable disk
  \end{lstlisting}
\item 
El output que ofrece \texttt{dmesg} contiene el contenido el contenido del archivo \texttt{/var/log/dmesg}, pero
sólo hasta cierto punto del proceso de boot del ordenador. De hecho \texttt{dmesg} está configurado en \texttt{/etc/init/dmesg.conf}
para copiar cuando arranca el proceso \texttt{init} los logs iniciales del proceso de booteo (en concreto 524288 bytes del
buffer donde se almacena esta información (conocido como \textit{kernel ring buffer}, porque una vez está lleno, la información
sigue escribiéndose sobreescribiendo la información más antigua). Este archivo de configuración contiene el siguiente código:

\begin{lstlisting}[style=BashInputStyle]
# dmesg - save kernel messages
#
# This task saves the initial kernel message log.

description	"save kernel messages"

start on runlevel [2345]

task
script
    savelog -q -p -c 5 /var/log/dmesg
    dmesg -s 524288 > /var/log/dmesg
    chgrp adm /var/log/dmesg
end script
\end{lstlisting}

Es en concreto la línea \texttt{dmesg -s 524288 > /var/log/dmesg} la que almacena la información en el archivo.

Aparte del archivo \texttt{/var/log/dmesg}, existen los siguientes:
\begin{itemize}
 \item \texttt{/var/log/kern.log} que contiene en las distros Debian la misma información que aporta \texttt{dmesg},
 pero con el timestamp de cada evento
 \item \texttt{/var/log/messages}, que contiene también el contenido de \texttt{/var/log/dmeseg} y contiene
 mensajes sobre eventos no críticos, al contrario que \texttt{/var/log/kern.log} que recoge mensajes sobre eventos
 críticos.
 \item \texttt{/var/log/syslog}, que es el log más general de las distros Debian. En particular, contiene a los dos logs
 mencionados.
\end{itemize}

\cite{dmesg} \cite{syslog}

 \end{enumerate}

\end{answer}
        
\section{Cuestión 5}
\textbf{\begin{enumerate}
         \item Ejecute el monitor de System Performance y muestre el resultado. Incluya capturas de pantalla y 
         comente la información que aparece.       
         \item Creen un recopilador de datos definido por el usuario (modo avanzado) que incluya tanto el contador de rendimiento
         como los datos de seguimiento:
         \begin{itemize}
          \item Todos los referentes al procesador, al proceso y al servicio web.
          \item Intervalo de muesta 15 segundos.
          \item Almacene el resultado en el direcotrio Escritorio$\backslash$logs
         \end{itemize}
	 Incluya las capturas de pantalla de cada paso.
         \end{enumerate}}
\begin{answer}
 \begin{enumerate}
  \item En la \ref{perf} se muestra una captura del Performance Monitor de Windows Server.
  
  \imagen{../images/51p3.jpeg}{Monitor de Rendimiento de Windows Server}{perf}{1.0}
  
  El monitor muestra para 100 segundos(este parámetro se le puede cambiar) 1 muestra por segundo en la que toma
  el valor de la utilización de la CPU en ese momento. Y va graficando todos los datos y obteniendo información sobre
  ellos, como la media, el máximo hasta el momento de utilización, y el mínimo. También se puede cambiar el \textit{Sampling Time},
  o cada cuánto recoge el monitor muestras. En la figura está configurado a 1 segundo.
  
  \item El proceso para crear el recopilador de datos se muestra en \ref{52}, \ref{53}, \ref{54}, \ref{55}, \ref{56},
  \ref{57}, \ref{58}, \ref{59}, \ref{510}
  
    \imagen{../images/52p3.jpeg}{Creamos un nuevo recopilador de datos}{52}{.8}
    \imagen{../images/53p3.jpeg}{Marcamos la opción de contador de rendimiento y de seguimiento}{53}{.8}
    \imagen{../images/54p3.jpeg}{Añadimos los proveedores relacionados con el procesador, el proceso, y el servicio web}{54}{.8}
    \imagen{../images/55p3.jpeg}{Añadimos los proveedores relacionados con el procesador, el proceso, y el servicio web}{55}{.8}
    \imagen{../images/56p3.jpeg}{Añadimos los proveedores relacionados con el procesador, el proceso, y el servicio web}{56}{.8}
    \imagen{../images/57p3.jpeg}{Clicamos en las proiedades del Contador de rendimiento, y nos aseguramos que sample interval=15}{57}{.8}
    \imagen{../images/58p3.jpeg}{Añadimos los de procesador, proceso y servicio web}{58}{.8}
    \imagen{../images/59p3.jpeg}{Clicamos en propiedades del recolector de datos}{59}{.8}
    \imagen{../images/510p3.jpeg}{En la pestaña Directory, cambiamos el directorio raíz a Escritorio$\backslash$ logs}{510}{.8}

 
 
 \end{enumerate}

 
\end{answer}
         
\section{Cuestión 6}
\textbf{Elija uno de los sistemas de monitorización descritos en este apartado e instálelo. Describa los pasos seguidos así
como posibles incidencias en la instalación que ha debido resolver. Monitorice uno o varios de sus servidores y presente
ejemplos de algunas medidas que considere significativas, explicando su significado y los valores reales observados.}
\begin{answer}
 Vamos a instalar Ganglia en Ubuntu Server.
 
 Ganglia está compuesto de un demonio de monitorización(\texttt{gmond}); de un demonio que recoge los datos de 
 todos los \texttt{gmond} instalados en los clientes, llamado \texttt{gmetad} y de un front-end con interfaz web
 que permite acceder a través de un servidor apache a los datos recogidos por Ganglia.
 
 Para la instalación de todo esto se ha ejecutado:
 \begin{lstlisting}[style=BashInputStyle]
  sudo apt-get install ganglia-monitor rrdtool gmetad ganglia-webfrontend
 \end{lstlisting}
 
 Y cuando ha pedido confirmación para reiniciar Apache, tal que en la \ref{reinitapache} hemos aceptado.
 
 \imagen{../images/62p3.jpeg}{Ganglia pide reiniciar Apache}{reinitapache}{1.0}
 
 Para activar el web front-end ha hecho falta ejecutar:
 \begin{lstlisting}[style=BashInputStyle]
  sudo cp /etc/ganglia-webfrontend/apache.conf /etc/apache2/sites-enabled/ganglia.conf
 \end{lstlisting}
 
 Para que esté disponible a través del servidor Apache.
 
 Y por último ha sido necesario iniciar los demonios de Ganglia y reiniciar Apache:
 \begin{lstlisting}[style=BashInputStyle]
sudo /etc/init.d/ganglia-monitor start
sudo /etc/init.d/gmetad start
sudo /etc/init.d/apache2 restart
 \end{lstlisting}

 Estos últimos pasos se corresponden a los de la \ref{ganglia}.
 
 \imagen{../images/63p3.jpeg}{Habilitando el front-end de \texttt{ganglia}}{ganglia}{1.0}
 
 Entramos en el navegador, a través de una IP que tenemos configurada en Ubuntu Server en la misma subred que el interfaz
 \texttt{vboxnet0} en la máquina host: \texttt{192.168.56.2}. Y entramos a la página: \texttt{192.168.56.2/ganglia}

 \texttt{ganglia} nos ofrece entre otras gráficas:
 \begin{itemize}
 \item La de la carga media del servidor: esto es número de procesos en el servidor por segundo, tal y como se muestra
 en la \ref{load}
 
   \imagen{../images/64p3.jpeg}{Carga de procesos en el servidor}{load}{.8}
 
 \item Tal y como se refleja en la \ref{net} se refleja la carga de la red: número de bytes transferidos (en azul)
 y número de bytes recibidos (en verde) por el servidor.
 
   \imagen{../images/65p3.jpeg}{Tráfico de red del servidor}{net}{.8}
 \end{itemize}
 
\end{answer}

\section{Cuestión 7}
\textbf{Diseñe un pequeño modelo de BBDDs para un problema de su elección e impleméntelo en MySQL. También puede emplear un sistema
de código abierto del que conozca su diseño de BBDDS. Plantee una combinación de consultas (al menos dos) que considere significativas
y explique los resultados obtenidos en su ``profile''. se valorará especialmente que sea capaz de introducir cambios en el 
diseño de tablas o en las consultas que mejoren los resultados y que sepa justificar la mejora.}
\begin{answer}
 
\end{answer}

\newpage
\printbibliography
\end{document}