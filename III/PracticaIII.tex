\documentclass[a4paper,10pt]{article}
% Símbolo del euro
\usepackage[gen]{eurosym}
% Codificación
\usepackage[utf8]{inputenc}
% Idioma
\usepackage[spanish]{babel} % English language/hyphenation
\selectlanguage{spanish}
% Hay que pelearse con babel-spanish para el alineamiento del punto decimal
\decimalpoint
\usepackage{dcolumn}
\newcolumntype{d}[1]{D{.}{\esperiod}{#1}}
\makeatletter
\addto\shorthandsspanish{\let\esperiod\es@period@code}
\makeatother
% To work with bibtex
%\usepackage{natbib}
\usepackage[backend=bibtex,style=numeric,sorting=none]{biblatex}
\bibliography{references}
\usepackage{longtable}
\usepackage{tabu}
\usepackage{supertabular}

\usepackage{multicol}
\newsavebox\ltmcbox

% Para algoritmos
%\usepackage{algorithm}
%\usepackage{algorithmic}
\usepackage{amsthm}
% Para matrices
\usepackage{amsmath}

% Símbolos matemáticos
\usepackage{amssymb}
\let\oldemptyset\emptyset
\let\emptyset\varnothing

% Hipervínculos
\usepackage{url}

\usepackage[section]{placeins} % Para gráficas en su sección.
\usepackage{mathpazo} % Use the Palatino font
\usepackage[T1]{fontenc} % Required for accented characters
\newenvironment{allintypewriter}{\ttfamily}{\par}
\setlength{\parindent}{0pt}
\parskip=8pt
\linespread{1.05} % Change line spacing here, Palatino benefits from a slight increase by default


% Imágenes
\usepackage{graphicx}
\usepackage{float}
\usepackage{caption}
\usepackage{wrapfig} % Allows in-line images

% Referencias
\usepackage{fncylab}
\labelformat{figure}{\textit{\figurename\space #1}}

\usepackage{hyperref}
\hypersetup{
  colorlinks   = true, %Colours links instead of ugly boxes
  urlcolor     = blue, %Colour for external hyperlinks
  linkcolor    = blue, %Colour of internal links
  citecolor   = red %Colour of citations
}

%Basado en: http://en.wikibooks.org/wiki/LaTeX/Theorems
\usepackage{amsthm}
\newtheorem*{mydef}{Definición}
\newtheorem{mydefn}{Definición}
\newtheorem{theorem}{Teorema}
\everymath{\displaystyle} % Displaystyle por defecto

% To include code
\usepackage{xcolor}
\usepackage{listings}

% code in bash style
\lstdefinestyle{BashInputStyle}{
  language=bash,
  basicstyle=\footnotesize\ttfamily,
  numberstyle=\tiny,
  numbersep=3pt,
  columns=fullflexible,
  backgroundcolor=\color{gray!5},
  showspaces=false,               % show spaces adding particular underscores
  showstringspaces=false,
  %xleftmargin=0.1\linewidth
}

% To change level of indentation
\newenvironment{answer}{%
\begin{list}{}{%
}%
\item[]}{\end{list}}


\makeatletter
\renewcommand{\@listI}{\itemsep=0pt} % Reduce the space between items in the itemize and enumerate environments and the bibliography
\newcommand{\imagent}[4]{
  \begin{wrapfigure}{#4}{0.7\textwidth}
    \begin{center}
    \includegraphics[width=0.7\textwidth]{#1}
    \end{center}
    \caption{#3}
    \label{#4}
  \end{wrapfigure}
}

\newcommand{\imagen}[4]{
  \begin{minipage}{\linewidth}
    \centering
    \includegraphics[width=#4\textwidth]{#1}
    \captionof{figure}{#2}
    \label{#3}
  \end{minipage} 
}

%Customize enumerate tag
\usepackage{enumitem}
%Sections don't get numbered
\setcounter{secnumdepth}{0}
\usepackage{fontspec}
\setmainfont{Arial}

\begin{document}
%%%%%%%%%%%%%%%%%%%%%%%%%%%%%%%%%%%%%%%%%
% University Assignment Title Page 
% LaTeX Template
% Version 1.0 (27/12/12)
%
% This template has been downloaded from:
% http://www.LaTeXTemplates.com
%
% Original author:
% WikiBooks (http://en.wikibooks.org/wiki/LaTeX/Title_Creation)
% Modified by: NCordon (https://github.com/NCordon)
%
% License:
% CC BY-NC-SA 3.0 (http://creativecommons.org/licenses/by-nc-sa/3.0/)
% 
% Instructions for using this template:
% This title page is capable of being compiled as is. This is not useful for 
% including it in another document. To do this, you have two options: 
%
% 1) Copy/paste everything between \begin{document} and \end{document} 
% starting at \begin{titlepage} and paste this into another LaTeX file where you 
% want your title page.
% OR
% 2) Remove everything outside the \begin{titlepage} and \end{titlepage} and 
% move this file to the same directory as the LaTeX file you wish to add it to. 
% Then add \input{./title_page_1.tex} to your LaTeX file where you want your
% title page.
%
%%%%%%%%%%%%%%%%%%%%%%%%%%%%%%%%%%%%%%%%%
\begin{titlepage}

\newcommand{\HRule}{\rule{\linewidth}{0.5mm}} % Defines a new command for the horizontal lines, change thickness here

\center % Center everything on the page
 
%----------------------------------------------------------------------------------------
%	HEADING SECTIONS
%----------------------------------------------------------------------------------------
\textsc{\LARGE Universidad de Granada}\\[1.5cm]
\textsc{\Large Ingeniería de Servidores}\\[0.5cm] 

%----------------------------------------------------------------------------------------
%	TITLE SECTION
%----------------------------------------------------------------------------------------
\bigskip
\HRule \\[0.4cm]
{ \huge \bfseries Práctica III}\\[0.4cm] % Title of your document
\HRule \\[1.5cm]
 
%----------------------------------------------------------------------------------------
%	AUTHOR SECTION
%----------------------------------------------------------------------------------------

\begin{minipage}{0.4\textwidth}
\begin{center} \large
\emph{Ignacio Cordón Castillo}\\
\end{center}
\end{minipage}

%----------------------------------------------------------------------------------------
%	LOGO SECTION
%----------------------------------------------------------------------------------------

\begin{center}
\includegraphics[width=9cm]{../images/ugr.jpg}
\end{center}
%----------------------------------------------------------------------------------------

\vspace{\fill}% Fill the rest of the page with whitespace
\large\today
\end{titlepage}  

\newpage
\tableofcontents
\newpage
% Examples of inclussion of images
%\imagent{ugr.jpg}{Logo de prueba}{ugr}
%\imagen{ugr.jpg}{Logo de prueba}{ugr2}{size relative to the \textwidth}

\section{Cuestión 1}
\textbf{\begin{enumerate}[label=\alph*.]
         \item En qué archivos se guarda registro de  los paquetes instalados en sistema con los gestores de paquetes de Ubuntu 
         y centOS. Durante la práctica 2 instaló LAMP como un único paquete o instalando cada componente diferenciado. Busque
         en el archivo de registro las líneas correspondientes a la instalación y preséntales.         
         \item En el directorio /var/log es común encontar archivos con extensiones en formato <numero>.gz.  Por  ejemplo,  
         .1.gz.  Explique  como  se  generan  estos  archivos  y  que relación guardan entre ellos
        \end{enumerate}}
\begin{answer}
 \begin{enumerate}[label=\alph*.] %[label=(\Alph*)] [label=(\roman*)]
  \item Para conocer los paquetes instalados en Ubuntu, basta consultar el archivo de órdenes introducidas en \texttt{apt-get},
  que se guarda en \texttt{/var/log/apt/history.log}, con la información
  de los paquetes que se instalaron, actualizaron o borraron, producto de ejecutar dichas órdenes.
  
  Ahora bien, como \texttt{apt-get} es un front-end (una intrerfaz) de \texttt{dpkg} en las distribuciones Debian, también
  podemos consultar el histórico de \texttt{dpkg} en \texttt{/var/log/dpkg.log}, que incluirá los paquetes instalados a través
  de \texttt{apt-get}, aunque también los instalados directamente a través de \texttt{dpkg}
  \cite{apt}
  
  El archivo que alamacena el histórico de paquetes instalados en CentOS, esto es, a través de \texttt{yum}, se encuentra en:
  \begin{center}\texttt{/var/log/yum.log}\end{center}
  \cite{yum}
  
  % Falta paquetes de LAMP
  \item Los archivos del tipo \texttt{daemon.log.1.gz}, \texttt{daemon.log.2.gz}, \ldots\\ corresponden a antiguos logs del comando
  \texttt{daemon} comprimidos. Análogamente, esto es extensible a los históricos de cualquier otro comando. Cada cierto tiempo, el archivo
  \texttt{.log} se renombra a un archivo de la forma \texttt{.log.<numero>} donde se van numerando de forma ascendente (a un número
  mayor, más reciente es el log), y el comando comenzará un nuevo histórico desde dicho momento. Pasado cierto tiempo, se comprimen dichos archivos, generando  \texttt{daemon.log.1.gz}, \texttt{daemon.log.2.gz}, \ldots; 
  de éste modo, los archivos ocupan menos espacio en disco duro, y aún están disponibles para su consulta.
  A este comportamiento, se le llama \textbf{Rotación de Logs}, y es el comando \texttt{logrotate}, programado a través del demonio
  \texttt{cron} quien se encarga de llevar a cabo dicha tarea de ir renombrando y comprimiento logs. Asimismo, dicho programa
  se puede encargar de crear nuevos logs una vez renombrados los antiguos, en función de si está especificado en su configuración
  o no. El archivo de configuración está disponible en \texttt{/etc/logrotate.conf}
  \cite{logs}
 \end{enumerate}

\end{answer}

\section{Cuestión 2}
\textbf{Dentro de /proc existe un archivo para el estado de los multidevice como es el caso de
nuestro /dev/md0 creado en la P1.\\
Vamos a monitorizar el proceso de sustituir un disco dañado por uno nuevo y cómo
podemos saber cuándo el sistema está preparado para continuar operando con
normalidad.\\
Indique los pasos que ha seguido, comandos empleados y significado de los mismos. Junto a cada comando, presente las líneas
del RAID que son significativas en cada paso: indicación de fallo, reemplazo, inicio y finalización de la reconstrucción del RAID.}
\begin{answer}
 
\end{answer}

\section{Cuestión 3}
\textbf{ Añada a la configuración de cron una tarea que se ejecute diariamente y que copie una vez al día el contenido del directorio
\~/codigo a \~/seguridad\$fecha donde \$fecha es la fecha actual del sistema (puede usar el comando
date). Otra tarea, se ejecutará una vez al mes y reunirá todos los directorios diarios creados para el mes pasado
en un archivo \~/seguridad/dirCodigo.<numero>.gz. Presente las líneas de configuración de cron afectadas, explicando
su significado. Si crea ficheros por lotes, presente y explique el código.}
\begin{answer}
 Creamos en \texttt{/etc/cron.daily} el siguiente script, de nombre \texttt{ej3}:
 
\begin{lstlisting}[style=BashInputStyle]
#!/bin/bash

if [[ -d ~/codigo ]]
then
    DIR=~/seguridad/$(date +%m-%d-%y)
    mkdir -p $DIR 2> /dev/null
    cp ~/codigo/* $DIR 
else
    mkdir -p ~/codigo
fi
\end{lstlisting}
  
  El script comprueba primero si existe el directorio \texttt{~/codigo}. Si no existe, lo crea. Si existe, crea una carpeta
  con el formato \texttt{mes-dia-año} de la fecha en que nos encontremos en \texttt{~/seguridad}, y copia el contenido de \texttt{~/codigo}
  en ella.
  
  Creamos en \texttt{/etc/cron.monthly} el siguiente script, de nombre \texttt{ej3}:
  
\begin{lstlisting}[style=BashInputStyle]
#!/bin/bash

let MONTH=$(date +%m)-1
N=0

if [[ MONTH -lt 10 ]]
then
    MONTH=0$MONTH
fi


for FILE in $(ls ~/seguridad/dirCodigo*.gz 2> /dev/null)
do
    ACTUAL=$(echo $FILE | egrep -o "[[:digit:]]+")
    let ACTUAL=$ACTUAL+1
    
    if [[ ACTUAL -gt N ]]
    then
	N=$ACTUAL
    fi
done

tar -czvf ~/seguridad/dirCodigo$N.gz ~/seguridad/${MONTH}* &> /dev/null
\end{lstlisting}
  
  El script halla el mes actual, y le resta uno. Como esta operación hará que el mes pierda el 0 inicial si lo llevaba, se lo añadimos
  si se trata de un mes menor a 10. Esto lo emplearemos para hallar los directorios del mes anterior posteriormente.
  El script halla a continuación el número por el que proseguir numerando los archivos \texttt{.gz}. Por ejemplo, si ya estaban
  creados los archivos \texttt{dirCodigo0.gz} y \texttt{dirCodigo1.gz}, el archivo que creará el script tendrá por nombre \texttt{dirCodigo2.gz}
  Por último el script empaqueta en el archivo \texttt{.gz} de número hallado anteriormente todos los directorios del mes anterior.
\end{answer}

\section{Cuestión 4}
\textbf{\begin{enumerate}
         \item Pruebe a ejecutar el comando, conectar un dispositivo USB y vuelva a ejecutar el comando. Copie y peque las líneas
         que hacen mención al dispositivo conectado (considere usar dmesg|tail).
         %??????????
         \item Explique las diferencias(si las hay) entre o consueltar el consultar el contenido del archivo /var/log/dmesg.
        \end{enumerate}}
\begin{answer}
 
\end{answer}
        
\section{Cuestión 5}
\textbf{\begin{enumerate}
         \item Ejecute el monitor de System Performance y muestre el resultado. Incluya capturas de pantalla y 
         comente la información que aparece.       
         \item Creen un recopilador de datos definido por el usuario (modo avanzado) que incluya tanto el contador de rendimiento
         como los datos de seguimiento:
         \begin{itemize}
          \item Todos los referentes al procesador, al proceso y al servicio web.
          \item Intervalo de muesta 15 segundos.
          \item Almacene el resultado en el direcotrio Escritorio$\backslash$logs
         \end{itemize}
	 Incluya las capturas de pantalla de cada paso.
         \end{enumerate}}
\begin{answer}
 
\end{answer}
         
\section{Cuestión 6}
\textbf{Elija uno de los sistemas de monitorización descritos en este apartado e instálelo. Describa los pasos seguidos así
como posibles incidencias en la isntalación que ha debido resolver. MOnitorice uno o varios de sus servidores y presente
ejemplos de algunas medidas que considere significativas, explicando su significado y los valores reales observados.}
\begin{answer}
 
\end{answer}

\section{Cuestión 7}
\textbf{Diseñe un pequeño modelo de BBDDs para un problema de su elección e impleméntelo en MySQL. También puede emplear un sistema
de código abierto del que conozca su diseño de BBDDS. Plantee una combinación de consultas (al menos dos) que considere significativas
y explique los resultados obtenidos en su ``profile''. se valorará especialmente que sea capaz de introducir cambios en el 
diseño de tablas o en las consultas que mejoren los resultados y que sepa justificar la mejora.}
\begin{answer}
 
\end{answer}

\newpage
\printbibliography
\end{document}