\documentclass[a4paper,10pt]{article}
% Símbolo del euro
\usepackage[gen]{eurosym}
% Codificación
\usepackage[utf8]{inputenc}
% Idioma
\usepackage[spanish]{babel} % English language/hyphenation
\selectlanguage{spanish}
% Hay que pelearse con babel-spanish para el alineamiento del punto decimal
\decimalpoint
\usepackage{dcolumn}
\newcolumntype{d}[1]{D{.}{\esperiod}{#1}}
\makeatletter
\addto\shorthandsspanish{\let\esperiod\es@period@code}
\makeatother
% To work with bibtex
%\usepackage{natbib}
\usepackage[backend=bibtex,style=numeric,sorting=none]{biblatex}
\renewcommand*{\newunitpunct}{\newline\space}


\bibliography{references}
\usepackage{longtable}
\usepackage{tabu}
\usepackage{supertabular}

\usepackage{multicol}
\newsavebox\ltmcbox

% Para algoritmos
%\usepackage{algorithm}
%\usepackage{algorithmic}
\usepackage{amsthm}
% Para matrices
\usepackage{amsmath}

% Símbolos matemáticos
\usepackage{amssymb}
\let\oldemptyset\emptyset
\let\emptyset\varnothing

% Hipervínculos
\usepackage{url}

\usepackage[section]{placeins} % Para gráficas en su sección.
\usepackage{mathpazo} % Use the Palatino font
\usepackage[T1]{fontenc} % Required for accented characters
\newenvironment{allintypewriter}{\ttfamily}{\par}
\setlength{\parindent}{0pt}
\parskip=8pt
\linespread{1.05} % Change line spacing here, Palatino benefits from a slight increase by default


% Imágenes
\usepackage{graphicx}
\usepackage{float}
\usepackage{caption}
\usepackage{wrapfig} % Allows in-line images

% Referencias
\usepackage{fncylab}
\labelformat{figure}{\textit{\figurename\space #1}}

\usepackage{hyperref}
\hypersetup{
  colorlinks   = true, %Colours links instead of ugly boxes
  urlcolor     = blue, %Colour for external hyperlinks
  linkcolor    = blue, %Colour of internal links
  citecolor   = red %Colour of citations
}

%Basado en: http://en.wikibooks.org/wiki/LaTeX/Theorems
\usepackage{amsthm}
\newtheorem*{mydef}{Definición}
\newtheorem{mydefn}{Definición}
\newtheorem{theorem}{Teorema}
\everymath{\displaystyle} % Displaystyle por defecto

% To include code
\usepackage{xcolor}
\usepackage{listings}

% code in bash style
\lstdefinestyle{BashInputStyle}{
  language=bash,
  basicstyle=\ttfamily,
  numberstyle=\tiny,
  numbersep=3pt,
  columns=fullflexible,
  backgroundcolor=\color{gray!10},
  showspaces=false,               % show spaces adding particular underscores
  showstringspaces=false,
  breaklines=true
  %xleftmargin=0.1\linewidth
}

% To change level of indentation
\newenvironment{answer}{%
\begin{list}{}{%
}%
\item[]}{\end{list}}


\makeatletter
\renewcommand{\@listI}{\itemsep=0pt} % Reduce the space between items in the itemize and enumerate environments and the bibliography
\newcommand{\imagent}[4]{
  \begin{wrapfigure}{#4}{0.7\textwidth}
    \begin{center}
    \includegraphics[width=0.7\textwidth]{#1}
    \end{center}
    \caption{#3}
    \label{#4}
  \end{wrapfigure}
}

\newcommand{\imagen}[4]{
  \begin{minipage}{\linewidth}
    \centering
    \includegraphics[width=#4\textwidth]{#1}
    \captionof{figure}{#2}
    \label{#3}
  \end{minipage} 
}

%Customize enumerate tag
\usepackage{enumitem}
%Sections don't get numbered
\setcounter{secnumdepth}{0}
\usepackage{fontspec}
\setmainfont{Arial}
\usepackage{geometry}
 \geometry{
 a4paper,
 total={210mm,297mm},
 left=30mm,
 right=30mm,
 top=25mm,
 bottom=25mm,
 }
 
\begin{document}
%%%%%%%%%%%%%%%%%%%%%%%%%%%%%%%%%%%%%%%%%
% University Assignment Title Page 
% LaTeX Template
% Version 1.0 (27/12/12)
%
% This template has been downloaded from:
% http://www.LaTeXTemplates.com
%
% Original author:
% WikiBooks (http://en.wikibooks.org/wiki/LaTeX/Title_Creation)
% Modified by: NCordon (https://github.com/NCordon)
%
% License:
% CC BY-NC-SA 3.0 (http://creativecommons.org/licenses/by-nc-sa/3.0/)
% 
% Instructions for using this template:
% This title page is capable of being compiled as is. This is not useful for 
% including it in another document. To do this, you have two options: 
%
% 1) Copy/paste everything between \begin{document} and \end{document} 
% starting at \begin{titlepage} and paste this into another LaTeX file where you 
% want your title page.
% OR
% 2) Remove everything outside the \begin{titlepage} and \end{titlepage} and 
% move this file to the same directory as the LaTeX file you wish to add it to. 
% Then add \input{./title_page_1.tex} to your LaTeX file where you want your
% title page.
%
%%%%%%%%%%%%%%%%%%%%%%%%%%%%%%%%%%%%%%%%%
\begin{titlepage}

\newcommand{\HRule}{\rule{\linewidth}{0.5mm}} % Defines a new command for the horizontal lines, change thickness here

\center % Center everything on the page
 
%----------------------------------------------------------------------------------------
%	HEADING SECTIONS
%----------------------------------------------------------------------------------------
\textsc{\LARGE Universidad de Granada}\\[1.5cm]
\textsc{\Large Ingeniería de Servidores}\\[0.5cm] 

%----------------------------------------------------------------------------------------
%	TITLE SECTION
%----------------------------------------------------------------------------------------
\bigskip
\HRule \\[0.4cm]
{ \huge \bfseries Práctica IV}\\[0.4cm] % Title of your document
\HRule \\[1.5cm]
 
%----------------------------------------------------------------------------------------
%	AUTHOR SECTION
%----------------------------------------------------------------------------------------

\begin{minipage}{0.4\textwidth}
\begin{center} \large
\emph{Ignacio Cordón Castillo}\\
\end{center}
\end{minipage}

%----------------------------------------------------------------------------------------
%	LOGO SECTION
%----------------------------------------------------------------------------------------

\begin{center}
\includegraphics[width=9cm]{../images/ugr.jpg}
\end{center}
%----------------------------------------------------------------------------------------

\vspace{\fill}% Fill the rest of the page with whitespace
\large\today
\end{titlepage}  

\newpage
\tableofcontents
\newpage
% Examples of inclussion of images
%\imagent{ugr.jpg}{Logo de prueba}{ugr}
%\imagen{ugr.jpg}{Logo de prueba}{ugr2}{size relative to the \textwidth}

\section{Cuestión 1}
\textbf{Instale Phoronix Suite, seleccione un benchmark de entre los disponibles, descárguelo y ejecútelo. Describa el 
  propósito del benchmark y su interés en el mismo. Explique razonadamente el significado de los resultados
  obtenidos}
\begin{answer}
 
\end{answer}

\section{Cuestión 2}
\textbf{Instale una de las aplicaciones de benchmark: SisoftSandra o Aida. Seleccione un benchmark y ejecútelo.
  Describa el propósito del benchmark y su interés en el mismo. Explique razonadamente el significado de los
  resultados obtenidos.}
\begin{answer}
 
\end{answer}

\section{Cuestión 3}
\textbf{Desarrolle un Benchmark.
 \begin{enumerate}[label=\alph*.]
  \item Diseño del Benchmark:\\
  Debe elegir un recurso IT sobre el que plantear su hipótesis. Se valorará
  especialmente que se opte por elementos de la infraestructura IT
  (servidores, recursos del SO, hardware, networking, etc.) frente a
  componentes de desarrollo de software (lenguajes de programación,
  algoritmos, etc.). En todo caso, el experimento debe poder reproducirse
  en el aula de prácticas. Por ello, no plantee experimentos que no puedan
  simularse en el aula.\\
  Explique el interés de la hipótesis elegida, así como el de los factores y
  niveles analizados. Como mínimo debe plantear un experimento de 1
  factor con 3 niveles.\\
  Describa la carga de trabajo elegida, la forma de registrarla o simularla y
  cómo alimenta el experimento.\\
  Describa las medidas tomadas y tratamiento estadístico aplicado para
  obtener las métricas. Justifique si la técnica empleada para la obtención
  de las medidas puede, o no, distorsionar los resultados observados.\\
  Finalmente, explique si se cumple o no la hipótesis de acuerdo con el
  factor de confianza elegido. El alumno deberá justificar, de manera
  especialmente detallada y razonada, la no obtención de resultados
  significativos para niveles de confianza estándar del 95\%.\\
  \item Programación del Benchmark:\\
  Programe el benchmark en un lenguaje de su elección. Se valorará que
  la tecnología elegida y el diseño sean adecuados a los factores a
  analizar.\\
  Mediante un organigrama, describa el funcionamiento del benchmark.\\
  El resultado de la ejecución del programa, será, como mínimo, un fichero
  de texto plano tabulado, en un formato estándar (csv, xml,\ldots) con
  las métricas obtenidas para los factores y niveles del experimento.\\
  El alumno podrá analizar estos resultados de dos formas:\\
  \begin{itemize}
   \item Integrando la lógica de análisis en el propio programa, mediante
    el uso de una librería estadística. Esta será la opción más valorada.
   \item Importando los resultados a una herramienta estadística de su elección
    y realizando el permitente análisis off-line.
  \end{itemize}
  En ambos casos, el alumno deberá explicar los resultados obtenidos.\\
  El benchmark debe funcionar de forma autónoma, reduciendo al máximo
  la necesidad de intervención del usuario. Por ejemplo, limitándola al
  cambio de dispositivo hardware. En todo caso, el programa debe interactuar
  con el usuario de forma sencilla y clara.\\
  \item Entregables:\\
  Las respuestas al apartado de diseño se realizarán en el habitual
  documento de memoria de prácticas.\\
  El alumno adjuntará, en un archivo comprimido y organizado en
  subdirectorios, todos los artefactos de desarrollo empleados en la
  creación del benchmark: código fuente debidamente comentado,
  documentación de diseño, archivos de datos con la carga de trabajo, etc.\\
  Además proporcionará un manual con la organización de los contenidos
  del archivo comprimido e instrucciones para la instalación y ejecución
  del programa.
 \end{enumerate}
}


 
 \begin{answer}
 
\end{answer}

\newpage
\printbibliography
\end{document}